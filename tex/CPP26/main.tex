\documentclass[sigplan,10pt,anonymous,review]{acmart}\settopmatter{printfolios=true,printccs=false,printacmref=false}
%% Rights management information.  This information is sent to you
%% when you complete the rights form.  These commands have SAMPLE
%% values in them; it is your responsibility as an author to replace
%% the commands and values with those provided to you when you
%% complete the rights form.
\setcopyright{acmlicensed}
\copyrightyear{2025}
\acmYear{2025}
\acmDOI{XXXXXXX.XXXXXXX}
%% These commands are for a PROCEEDINGS abstract or paper.
\acmConference[Conference acronym 'XX]{Make sure to enter the correct
  conference title from your rights confirmation email}{June 03--05,
  2018}{Woodstock, NY}
%%
%%  Uncomment \acmBooktitle if the title of the proceedings is different
%%  from ``Proceedings of ...''!
%%
%%\acmBooktitle{Woodstock '18: ACM Symposium on Neural Gaze Detection,
%%  June 03--05, 2018, Woodstock, NY}
\acmISBN{978-1-4503-XXXX-X/2018/06}

%%% Packages %%%%%%%%%%%%%%%%%%%%%%%%%%%%%%%%%%%%%%%%%%%%
\usepackage{xspace}
%%%%%%%%%%%%%%%%%%%%%%%%%%%%%%%%%%%%%%%%%%%%%%%%%%%%%%%%%

%%% Macros %%%%%%%%%%%%%%%%%%%%%%%%%%%%%%%%%%%%%%%%%%%%%%
\newcommand{\CA}{\textsc{Cubical Agda}\xspace}
%%%%%%%%%%%%%%%%%%%%%%%%%%%%%%%%%%%%%%%%%%%%%%%%%%%%%%%%%

\begin{document}

%%
%% The "title" command has an optional parameter,
%% allowing the author to define a "short title" to be used in page headers.
\title{Can Natural Models Simplify The Metatheory of Type Theory in Cubical Agda?}

%%
%% The "author" command and its associated commands are used to define
%% the authors and their affiliations.
%% Of note is the shared affiliation of the first two authors, and the
%% "authornote" and "authornotemark" commands
%% used to denote shared contribution to the research.
\author{Liang-Ting Chen}
\affiliation{%
  \institution{Institute of Information Science, Academia Sinica}
  \city{Taipei}
  \country{Taiwan}}
\email{liangtingchen@as.edu.tw}

\author{Fredrik Nordvall Forsberg}
\affiliation{%
  \institution{Department of Computer and Information Sciences, University of Strathclyde}
  \city{Glasgow}
  \country{United Kingdom}}

\author{Tzu-Chun Tsai}
\affiliation{%
  \institution{Institute of Information Science, Academia Sinica}
  \city{Taipei}
  \country{Taiwan}}
%\affiliation{%
%  \institution{Institute for Logic, Language and Computation, University of Amsterdam}
%  \city{Amsterdam}
%  \country{Netherlands}}

%%
%% By default, the full list of authors will be used in the page
%% headers. Often, this list is too long, and will overlap
%% other information printed in the page headers. This command allows
%% the author to define a more concise list
%% of authors' names for this purpose.
%\renewcommand{\shortauthors}{Trovato et al.}

%%
%% The abstract is a short summary of the work to be presented in the
%% article.
\begin{abstract}

  We formalise type theory in type theory, using Awodey's notion of \emph{natural model} of type theory.
  The initial natural model can be represented using quotient inductive-inductive-recursive types in the proof assistant \CA.
Using natural models leads to fewer instances of so-called ``transport hell'' in the syntax of type theory and its elimination rules.
To show that the approach is feasible, we formalise some meta-properties, including the standard % and logical predicate
interpretation, normalisation by evaluation for typed terms, and strictification constructions.
Since our formalisation is carried out using \CA's built-in support for quotient types, all our constructions compute.
However, the use of transports along equations often reappeared when developing more sophisticated metatheory.
Hence it is still a considerable struggle to develop the metatheory of type theory using welltyped terms with current proof assistant technology, and the effort is about the same using natural models or not.

%   This paper presents a formalisation of a core type theory and its metatheory within \CA.
% The type theory features $\Pi$-types, an inductive type of Booleans, and a universe.
% We formalise key meta-properties, including standard and logical predicate interpretations, canonicity, and normalisation by evaluation for typed terms.

% The type theory is defined intrinsically as a representable map of presheaves using quotient inductive-inductive-recursive types.
% Moreover, type theory and its elimination are defined with dependent identities without transports.
% This work is the first formalization of type theory that uses \CA's built-in support for quotient types, without using Licata's workaround and any rewriting rules, whereas its proofs of canonicity and normalisation are computational.
\end{abstract}

%%
%% The code below is generated by the tool at http://dl.acm.org/ccs.cfm.
%% Please copy and paste the code instead of the example below.
%%
\begin{CCSXML}
<ccs2012>
 <concept>
  <concept_id>00000000.0000000.0000000</concept_id>
  <concept_desc>Do Not Use This Code, Generate the Correct Terms for Your Paper</concept_desc>
  <concept_significance>500</concept_significance>
 </concept>
 <concept>
  <concept_id>00000000.00000000.00000000</concept_id>
  <concept_desc>Do Not Use This Code, Generate the Correct Terms for Your Paper</concept_desc>
  <concept_significance>300</concept_significance>
 </concept>
 <concept>
  <concept_id>00000000.00000000.00000000</concept_id>
  <concept_desc>Do Not Use This Code, Generate the Correct Terms for Your Paper</concept_desc>
  <concept_significance>100</concept_significance>
 </concept>
 <concept>
  <concept_id>00000000.00000000.00000000</concept_id>
  <concept_desc>Do Not Use This Code, Generate the Correct Terms for Your Paper</concept_desc>
  <concept_significance>100</concept_significance>
 </concept>
</ccs2012>
\end{CCSXML}

\ccsdesc[500]{Do Not Use This Code~Generate the Correct Terms for Your Paper}
\ccsdesc[300]{Do Not Use This Code~Generate the Correct Terms for Your Paper}
\ccsdesc{Do Not Use This Code~Generate the Correct Terms for Your Paper}
\ccsdesc[100]{Do Not Use This Code~Generate the Correct Terms for Your Paper}

%%
%% Keywords. The author(s) should pick words that accurately describe
%% the work being presented. Separate the keywords with commas.
\keywords{Do, Not, Us, This, Code, Put, the, Correct, Terms, for,
  Your, Paper}
%% A "teaser" image appears between the author and affiliation
%% information and the body of the document, and typically spans the
%% page.

%\received{20 February 2007}
%\received[revised]{12 March 2009}
%\received[accepted]{5 June 2009}

%%
%% This command processes the author and affiliation and title
%% information and builds the first part of the formatted document.
\maketitle
\bibliographystyle{ACM-Reference-Format}

\section{Introduction}
% FNF (Fri 5 Sep)

Transport hell = transports appear inside terms that you actually want to study, hence you need to do a lot of reasoning about transport rather than the actual objects of interest.


\section{Setting and metatheory}
% FNF (Sun 7 Sep)

Cubical Agda with UIP

QIIRTs

\section{Type theory as a natural model}
% LTC (Sun 7 Sep)

\subsection{Natural models}

\subsection{The Syntax of Type Theory as a Quotient Inductive-Inductive-Recursive Type}

QIIRT def of the syntax

\subsection{Recursion and Elimination Principles}

\subsection{Strictification}

\section{Metatheory}
% LTC (Tue 9 Sep)

\subsection{Standard Model}

% Argue for relative consistency?

\subsection{Normalisation by Evaluation}

% (for SC)

\section{Comparison with Other Approaches}
% FNF (Tue 9 Sep)

Compared to QIIT:

\begin{itemize}
\item Fewer transports in the syntax, but they tend to come back in concrete models
\item Strictification orthogonal
\end{itemize}

Compared to untyped version:
\begin{itemize}
\item Untyped version might still be easiest to work with, with current proof assistant technology
\end{itemize}

\section{Discussion}
% LTC (at first)

Practical considerations (eg NBE computes, support for interleaved mutual definitions)

Support from proof assistant (eg safety guarantees)

Easier in OTT-based proof assistant

\end{document}



%\begin{acks}
%To Robert, for the bagels and explaining CMYK and color spaces.
%\end{acks}
