
\documentclass[a4paper,UKenglish,numberwithinsect,cleveref,thm-restate]{lipics-v2021}
%This is a template for producing LIPIcs articles. 
%See lipics-v2021-authors-guidelines.pdf for further information.
%for A4 paper format use option "a4paper", for US-letter use option "letterpaper"
%for british hyphenation rules use option "UKenglish", for american hyphenation rules use option "USenglish"
%for section-numbered lemmas etc., use "numberwithinsect"
%for enabling cleveref support, use "cleveref"
%for enabling autoref support, use "autoref"
%for anonymousing the authors (e.g. for double-blind review), add "anonymous"
%for enabling thm-restate support, use "thm-restate"

%\pdfoutput=1 %uncomment to ensure pdflatex processing (mandatatory e.g. to submit to arXiv)
%\hideLIPIcs  %uncomment to remove references to LIPIcs series (logo, DOI, ...), e.g. when preparing a pre-final version to be uploaded to arXiv or another public repository

%\graphicspath{{./graphics/}}%helpful if your graphic files are in another directory

\bibliographystyle{plainurl}

\title{Type theories as quotient inductive-recursive types}
\author{Liang-Ting Chen\footnote{Corresponding author; authors are listed in alphabetical order.}}{Institute of Information Science, Academia Sinica, Taiwan \and \url{http://l-tchen.github.io}}{ltchen@iis.sinica.edu.tw}{https://orcid.org/0000-0002-3250-1331}{Supported by the National Science and Technology Council of Taiwan under grant NSTC [funding].}
\author{Tzu-Chun Tsai}{Institute of Information Science, Academia Sinica, Taiwan}{gene0905@icloud.com}{}{Supported by the National Science and Technology Council of Taiwan under grant NSTC 112-2221-E-001-003-MY3.}
\authorrunning{L.-T.~Chen and T.-C.~Tsai}
\Copyright{Liang-Ting Chen and Tzu-Chun Tsai}
\ccsdesc[500]{Theory of computation~Type theory}
\keywords{inductive-inductive types, quotient inductive types, inductive-recursive types, substitution calculus, category with families}
\relatedversion{} %optional, e.g. full version hosted on arXiv, HAL, or other respository/website
%\relatedversiondetails[linktext={opt. text shown instead of the URL}, cite=DBLP:books/mk/GrayR93]{Classification (e.g. Full Version, Extended Version, Previous Version}{URL to related version} %linktext and cite are optional

\supplement{The formal development is hosted at the GitHub repository: \url{https://github.com/genetsai95/DTT-QIIRT}.}%optional, e.g. related research data, source code, ... hosted on a repository like zenodo, figshare, GitHub, ...
%\supplementdetails[linktext={opt. text shown instead of the URL}, cite=DBLP:books/mk/GrayR93, subcategory={Description, Subcategory}, swhid={Software Heritage Identifier}]{General Classification (e.g. Software, Dataset, Model, ...)}{URL to related version} %linktext, cite, and subcategory are optional
\acknowledgements{We'd like to thank Fredrik Nordvall Forsberg, Hsiang-Shang Ko, and Meven Lennon-Bertrand.}%optional

\nolinenumbers %uncomment to disable line numbering

%Editor-only macros:: begin (do not touch as author)%%%%%%%%%%%%%%%%%%%%%%%%%%%%%%%%%%
\EventEditors{John Q. Open and Joan R. Access}
\EventNoEds{2}
\EventLongTitle{42nd Conference on Very Important Topics (CVIT 2016)}
\EventShortTitle{CVIT 2016}
\EventAcronym{CVIT}
\EventYear{2016}
\EventDate{December 24--27, 2016}
\EventLocation{Little Whinging, United Kingdom}
\EventLogo{}
\SeriesVolume{42}
\ArticleNo{23}
%%%%%%%%%%%%%%%%%%%%%%%%%%%%%%%%%%%%%%%%%%%%%%%%%%%%%%

\usepackage{todonotes}
\usepackage{manfnt}
\newcommand{\danger}{\marginpar[\hfill\dbend]{\dbend\hfill}}
\newcommand{\LT}[1]{\todo[inline,author={L-T},caption={}]{#1}}

\usepackage{xspace,bbold}
\usepackage{mathtools}

\newcommand{\mathsc}[1]{\textnormal{\textsc{#1}}}

\newcommand{\Agda}{\textsc{Agda}\xspace}
\newcommand{\Coq}{\textsc{Coq}\xspace}
\newcommand{\Lean}{\textsc{Lean}\xspace}
\newcommand{\Dedukti}{\textsf{Dedukti}\xspace}
\newcommand{\AxiomK}{Axiom~\textsf{K}\xspace}

\newcommand{\xto}[1]{\xrightarrow{#1}}
\newcommand{\comp}{\circ} % function composition
  
\mathchardef\mhyphen="2D % hyphen in mathmode

\newcommand{\Ty}{\ensuremath{\mathsf{Ty}}\xspace}
\newcommand{\Tm}{\ensuremath{\mathsf{Tm}}\xspace}
\newcommand{\Ctx}{\ensuremath{\mathsf{Ctx}}\xspace}
\newcommand{\Tel}{\ensuremath{\mathsf{Tel}}\xspace}
\newcommand{\Sub}{\ensuremath{\mathsf{Sub}}\xspace}
\newcommand{\Set}{\ensuremath{\mathsf{Set}}\xspace}
\newcommand{\El}{\ensuremath{\mathsf{El}}\xspace}
\newcommand{\Id}[3]{\ensuremath{#2 \;=_{#1}\;#3}\xspace}
\newcommand{\elim}{\ensuremath{\mathsf{elim}}\xspace}
\newcommand{\implicit}[1]{{\color{gray}\{#1\}}}

\newcommand{\IR}{\ensuremath{\mathsf{QIIR}}\xspace}
\newcommand{\I}{\ensuremath{\mathsf{QII}}\xspace}
\newcommand{\emptytel}{\ensuremath{\cdot}}

\newcommand{\reduce}{\mathbin{\Rightarrow}}
\newcommand\dplus{\ensuremath{\mathbin{+\mkern-10mu+}}}

\newcommand{\transp}{\ensuremath{\mathsf{transport}}\xspace}
\newcommand{\alert}[1]{{\color{red}#1}}

% Shamelessly copied from the HoTT book

\newcommand{\nameless}{\mathord{\hspace{1pt}\underline{\hspace{1ex}}\hspace{1pt}}}

%%% Function extensionality
\newcommand{\funext}{\mathsf{funext}}
\newcommand{\happly}{\mathsf{happly}}


%%%% MACROS FOR NOTATION %%%%
% Use these for any notation where there are multiple options.

\newcommand{\blank}{\mathord{\hspace{1pt}\text{--}\hspace{1pt}}} 
%%% Definitional equality (used infix) %%%
\newcommand{\jdeq}{\equiv}      % An equality judgment
\let\judgeq\jdeq
%\newcommand{\defeq}{\coloneqq}  % An equality currently being defined
\newcommand{\defeq}{\vcentcolon\equiv}  % A judgmental equality currently being defined

%%% Term being defined
\newcommand{\define}[1]{\textbf{#1}}

%%% Vec (for example)

\newcommand{\Vect}{\ensuremath{\mathsf{Vec}}}
\newcommand{\Fin}{\ensuremath{\mathsf{Fin}}}
\newcommand{\fmax}{\ensuremath{\mathsf{fmax}}}
\newcommand{\seq}[1]{\langle #1\rangle}

%%% Dependent products %%%
\def\prdsym{\textstyle\prod}
%% Call the macro like \prd{x,y:A}{p:x=y} with any number of
%% arguments.  Make sure that whatever comes *after* the call doesn't
%% begin with an open-brace, or it will be parsed as another argument.
\makeatletter
% Currently the macro is configured to produce
%     {\textstyle\prod}(x:A) \; {\textstyle\prod}(y:B),{\ }
% in display-math mode, and
%     \prod_{(x:A)} \prod_{y:B}
% in text-math mode.
% \def\prd#1{\@ifnextchar\bgroup{\prd@parens{#1}}{%
%     \@ifnextchar\sm{\prd@parens{#1}\@eatsm}{%
%         \prd@noparens{#1}}}}
\def\prd#1{\@ifnextchar\bgroup{\prd@parens{#1}}{%
    \@ifnextchar\sm{\prd@parens{#1}\@eatsm}{%
    \@ifnextchar\prd{\prd@parens{#1}\@eatprd}{%
    \@ifnextchar\;{\prd@parens{#1}\@eatsemicolonspace}{%
    \@ifnextchar\\{\prd@parens{#1}\@eatlinebreak}{%
    \@ifnextchar\narrowbreak{\prd@parens{#1}\@eatnarrowbreak}{%
      \prd@noparens{#1}}}}}}}}
\def\prd@parens#1{\@ifnextchar\bgroup%
  {\mathchoice{\@dprd{#1}}{\@tprd{#1}}{\@tprd{#1}}{\@tprd{#1}}\prd@parens}%
  {\@ifnextchar\sm%
    {\mathchoice{\@dprd{#1}}{\@tprd{#1}}{\@tprd{#1}}{\@tprd{#1}}\@eatsm}%
    {\mathchoice{\@dprd{#1}}{\@tprd{#1}}{\@tprd{#1}}{\@tprd{#1}}}}}
\def\@eatsm\sm{\sm@parens}
\def\prd@noparens#1{\mathchoice{\@dprd@noparens{#1}}{\@tprd{#1}}{\@tprd{#1}}{\@tprd{#1}}}
% Helper macros for three styles
\def\lprd#1{\@ifnextchar\bgroup{\@lprd{#1}\lprd}{\@@lprd{#1}}}
\def\@lprd#1{\mathchoice{{\textstyle\prod}}{\prod}{\prod}{\prod}({\textstyle #1})\;}
\def\@@lprd#1{\mathchoice{{\textstyle\prod}}{\prod}{\prod}{\prod}({\textstyle #1}),\ }
\def\tprd#1{\@tprd{#1}\@ifnextchar\bgroup{\tprd}{}}
\def\@tprd#1{\mathchoice{{\textstyle\prod_{(#1)}}}{\prod_{(#1)}}{\prod_{(#1)}}{\prod_{(#1)}}}
\def\dprd#1{\@dprd{#1}\@ifnextchar\bgroup{\dprd}{}}
\def\@dprd#1{\prod_{(#1)}\,}
\def\@dprd@noparens#1{\prod_{#1}\,}

% Look through spaces and linebreaks
\def\@eatnarrowbreak\narrowbreak{%
  \@ifnextchar\prd{\narrowbreak\@eatprd}{%
    \@ifnextchar\sm{\narrowbreak\@eatsm}{%
      \narrowbreak}}}
\def\@eatlinebreak\\{%
  \@ifnextchar\prd{\\\@eatprd}{%
    \@ifnextchar\sm{\\\@eatsm}{%
      \\}}}
\def\@eatsemicolonspace\;{%
  \@ifnextchar\prd{\;\@eatprd}{%
    \@ifnextchar\sm{\;\@eatsm}{%
      \;}}}

%%% Lambda abstractions.
% Each variable being abstracted over is a separate argument.  If
% there is more than one such argument, they *must* be enclosed in
% braces.  Arguments can be untyped, as in \lam{x}{y}, or typed with a
% colon, as in \lam{x:A}{y:B}. In the latter case, the colons are
% automatically noticed and (with current implementation) the space
% around the colon is reduced.  You can even give more than one variable
% the same type, as in \lam{x,y:A}.
\def\lam#1{{\lambda}\@lamarg#1:\@endlamarg\@ifnextchar\bgroup{.\,\lam}{.\,}}
\def\@lamarg#1:#2\@endlamarg{\if\relax\detokenize{#2}\relax #1\else\@lamvar{\@lameatcolon#2},#1\@endlamvar\fi}
\def\@lamvar#1,#2\@endlamvar{(#2\,{:}\,#1)}
% \def\@lamvar#1,#2{{#2}^{#1}\@ifnextchar,{.\,{\lambda}\@lamvar{#1}}{\let\@endlamvar\relax}}
\def\@lameatcolon#1:{#1}
\let\lamt\lam
% This version silently eats any typing annotation.
\def\lamu#1{{\lambda}\@lamuarg#1:\@endlamuarg\@ifnextchar\bgroup{.\,\lamu}{.\,}}
\def\@lamuarg#1:#2\@endlamuarg{#1}

%%% Dependent products written with \forall, in the same style
\def\fall#1{\forall (#1)\@ifnextchar\bgroup{.\,\fall}{.\,}}

%%% Existential quantifier %%%
\def\exis#1{\exists (#1)\@ifnextchar\bgroup{.\,\exis}{.\,}}

%%% Dependent sums %%%
\def\smsym{\textstyle\sum}
% Use in the same way as \prd
\def\sm#1{\@ifnextchar\bgroup{\sm@parens{#1}}{%
    \@ifnextchar\prd{\sm@parens{#1}\@eatprd}{%
    \@ifnextchar\sm{\sm@parens{#1}\@eatsm}{%
    \@ifnextchar\;{\sm@parens{#1}\@eatsemicolonspace}{%
    \@ifnextchar\\{\sm@parens{#1}\@eatlinebreak}{%
    \@ifnextchar\narrowbreak{\sm@parens{#1}\@eatnarrowbreak}{%
        \sm@noparens{#1}}}}}}}}
\def\sm@parens#1{\@ifnextchar\bgroup%
  {\mathchoice{\@dsm{#1}}{\@tsm{#1}}{\@tsm{#1}}{\@tsm{#1}}\sm@parens}%
  {\@ifnextchar\prd%
    {\mathchoice{\@dsm{#1}}{\@tsm{#1}}{\@tsm{#1}}{\@tsm{#1}}\@eatprd}%
    {\mathchoice{\@dsm{#1}}{\@tsm{#1}}{\@tsm{#1}}{\@tsm{#1}}}}}
\def\@eatprd\prd{\prd@parens}
\def\sm@noparens#1{\mathchoice{\@dsm@noparens{#1}}{\@tsm{#1}}{\@tsm{#1}}{\@tsm{#1}}}
\def\lsm#1{\@ifnextchar\bgroup{\@lsm{#1}\lsm}{\@@lsm{#1}}}
\def\@lsm#1{\mathchoice{{\textstyle\sum}}{\sum}{\sum}{\sum}({\textstyle #1})\;}
\def\@@lsm#1{\mathchoice{{\textstyle\sum}}{\sum}{\sum}{\sum}({\textstyle #1}),\ }
\def\tsm#1{\@tsm{#1}\@ifnextchar\bgroup{\tsm}{}}
\def\@tsm#1{\mathchoice{{\textstyle\sum_{(#1)}}}{\sum_{(#1)}}{\sum_{(#1)}}{\sum_{(#1)}}}
\def\dsm#1{\@dsm{#1}\@ifnextchar\bgroup{\dsm}{}}
\def\@dsm#1{\sum_{(#1)}\,}
\def\@dsm@noparens#1{\sum_{#1}\,}

% Other notations related to dependent sums
%\let\setof\Set    % from package 'braket', write \setof{ x:A | P(x) }.
\newcommand{\pair}{\ensuremath{\mathsf{pair}}\xspace}
\newcommand{\tup}[2]{(#1,#2)}
\newcommand{\proj}[1]{\ensuremath{\mathsf{pr}_{#1}}\xspace}
\newcommand{\fst}{\ensuremath{\proj1}\xspace}
\newcommand{\snd}{\ensuremath{\proj2}\xspace}
\newcommand{\ac}{\ensuremath{\mathsf{ac}}\xspace} % not needed in symbol index

%%% recursor and induction
\newcommand{\rec}[1]{\mathsf{rec}_{#1}}
\newcommand{\ind}[1]{\mathsf{ind}_{#1}}
\newcommand{\indid}[1]{\ind{=_{#1}}} % (Martin-Lof) path induction principle for identity types
\newcommand{\indidb}[1]{\ind{=_{#1}}'} % (Paulin-Mohring) based path induction principle for identity types

%%% Uniqueness principles
\newcommand{\uniq}[1]{\mathsf{uniq}_{#1}}

% Paths in pairs
\newcommand{\pairpath}{\ensuremath{\mathsf{pair}^{\mathord{=}}}\xspace}
% \newcommand{\projpath}[1]{\proj{#1}^{\mathord{=}}}
\newcommand{\projpath}[1]{\ensuremath{\apfunc{\proj{#1}}}\xspace}
\newcommand{\pairct}{\ensuremath{\mathsf{pair}^{\mathord{\ct}}}\xspace}

%%% For quotients %%%
%\newcommand{\pairr}[1]{{\langle #1\rangle}}
\newcommand{\pairr}[1]{{\mathopen{}(#1)\mathclose{}}}
\newcommand{\Pairr}[1]{{\mathopen{}\left(#1\right)\mathclose{}}}

\newcommand{\im}{\ensuremath{\mathsf{im}}} % the image

%%% 2D path operations
\newcommand{\leftwhisker}{\mathbin{{\ct}_{\mathsf{l}}}}  % was \ell
\newcommand{\rightwhisker}{\mathbin{{\ct}_{\mathsf{r}}}} % was r
\newcommand{\hct}{\star}

%%% Identity types %%%
\newcommand{\idsym}{{=}}
\newcommand{\id}[3][]{\ensuremath{#2 =_{#1} #3}\xspace}
\newcommand{\idtype}[3][]{\ensuremath{\mathsf{Id}_{#1}(#2,#3)}\xspace}
\newcommand{\idtypevar}[1]{\ensuremath{\mathsf{Id}_{#1}}\xspace}
% A propositional equality currently being defined
\newcommand{\defid}{\coloneqq}

%%% Dependent paths
\newcommand{\dpath}[4]{#3 =^{#1}_{#2} #4}

%%% Reflexivity terms %%%
% \newcommand{\reflsym}{{\mathsf{refl}}}
\newcommand{\refl}[1]{\ensuremath{\mathsf{refl}_{#1}}\xspace}

%%% Path concatenation (used infix, in diagrammatic order) %%%
\newcommand{\ct}{%
  \mathchoice{\mathbin{\raisebox{0.5ex}{$\displaystyle\centerdot$}}}%
             {\mathbin{\raisebox{0.5ex}{$\centerdot$}}}%
             {\mathbin{\raisebox{0.25ex}{$\scriptstyle\,\centerdot\,$}}}%
             {\mathbin{\raisebox{0.1ex}{$\scriptscriptstyle\,\centerdot\,$}}}
}

%%% Path reversal %%%
\newcommand{\opp}[1]{\mathord{{#1}^{-1}}}
\let\rev\opp

%%% Coherence paths %%%
\newcommand{\ctassoc}{\mathsf{assoc}} % associativity law

%%% Transport (covariant) %%%
\newcommand{\trans}[2]{\ensuremath{{#1}_{*}\mathopen{}\left({#2}\right)\mathclose{}}\xspace}
\let\Trans\trans
%\newcommand{\Trans}[2]{\ensuremath{{#1}_{*}\left({#2}\right)}\xspace}
\newcommand{\transf}[1]{\ensuremath{{#1}_{*}}\xspace} % Without argument
%\newcommand{\transport}[2]{\ensuremath{\mathsf{transport}_{*} \: {#2}\xspace}}
\newcommand{\transfib}[3]{\ensuremath{\mathsf{transport}^{#1}({#2},{#3})\xspace}}
\newcommand{\Transfib}[3]{\ensuremath{\mathsf{transport}^{#1}\Big(#2,\, #3\Big)\xspace}}
\newcommand{\transfibf}[1]{\ensuremath{\mathsf{transport}^{#1}\xspace}}

%%% 2D transport
\newcommand{\transtwo}[2]{\ensuremath{\mathsf{transport}^2\mathopen{}\left({#1},{#2}\right)\mathclose{}}\xspace}

%%% Constant transport
\newcommand{\transconst}[3]{\ensuremath{\mathsf{transportconst}}^{#1}_{#2}(#3)\xspace}
\newcommand{\transconstf}{\ensuremath{\mathsf{transportconst}}\xspace}

%%% Map on paths %%%
\newcommand{\mapfunc}[1]{\ensuremath{\mathsf{ap}_{#1}}\xspace} % Without argument
\newcommand{\map}[2]{\ensuremath{{#1}\mathopen{}\left({#2}\right)\mathclose{}}\xspace}
\let\Ap\map
%\newcommand{\Ap}[2]{\ensuremath{{#1}\left({#2}\right)}\xspace}
\newcommand{\mapdepfunc}[1]{\ensuremath{\mathsf{apd}_{#1}}\xspace} % Without argument
% \newcommand{\mapdep}[2]{\ensuremath{{#1}\llparenthesis{#2}\rrparenthesis}\xspace}
\newcommand{\mapdep}[2]{\ensuremath{\mapdepfunc{#1}\mathopen{}\left(#2\right)\mathclose{}}\xspace}
\let\apfunc\mapfunc
\let\ap\map
\let\apdfunc\mapdepfunc
\let\apd\mapdep

%%% 2D map on paths
\newcommand{\aptwofunc}[1]{\ensuremath{\mathsf{ap}^2_{#1}}\xspace}
\newcommand{\aptwo}[2]{\ensuremath{\aptwofunc{#1}\mathopen{}\left({#2}\right)\mathclose{}}\xspace}
\newcommand{\apdtwofunc}[1]{\ensuremath{\mathsf{apd}^2_{#1}}\xspace}
\newcommand{\apdtwo}[2]{\ensuremath{\apdtwofunc{#1}\mathopen{}\left(#2\right)\mathclose{}}\xspace}

%%% Identity functions %%%
\newcommand{\idfunc}[1][]{\ensuremath{\mathsf{id}_{#1}}\xspace}

%%% Other meanings of \sim
\newcommand{\bisim}{\sim}       % bisimulation
\newcommand{\eqr}{\sim}         % an equivalence relation

%%% Equivalence types %%%
\newcommand{\eqv}[2]{\ensuremath{#1 \simeq #2}\xspace}
\newcommand{\eqvspaced}[2]{\ensuremath{#1 \;\simeq\; #2}\xspace}
\newcommand{\eqvsym}{\simeq}    % infix symbol
\newcommand{\texteqv}[2]{\ensuremath{\mathsf{Equiv}(#1,#2)}\xspace}
\newcommand{\isequiv}{\ensuremath{\mathsf{isequiv}}}
\newcommand{\qinv}{\ensuremath{\mathsf{qinv}}}
\newcommand{\ishae}{\ensuremath{\mathsf{ishae}}}
\newcommand{\linv}{\ensuremath{\mathsf{linv}}}
\newcommand{\rinv}{\ensuremath{\mathsf{rinv}}}
\newcommand{\biinv}{\ensuremath{\mathsf{biinv}}}
\newcommand{\lcoh}[3]{\mathsf{lcoh}_{#1}(#2,#3)}
\newcommand{\rcoh}[3]{\mathsf{rcoh}_{#1}(#2,#3)}
\newcommand{\hfib}[2]{{\mathsf{fib}}_{#1}(#2)}

%%% Map on total spaces %%%
\newcommand{\total}[1]{\ensuremath{\mathsf{total}(#1)}}

%%% Universe types %%%
%\newcommand{\type}{\ensuremath{\mathsf{Type}}\xspace}
\newcommand{\UU}{\ensuremath{\mathcal{U}}\xspace}
% Universes of truncated types
\newcommand{\typele}[1]{\ensuremath{{#1}\text-\mathsf{Type}}\xspace}
\newcommand{\typeleU}[1]{\ensuremath{{#1}\text-\mathsf{Type}_\UU}\xspace}
\newcommand{\typelep}[1]{\ensuremath{{(#1)}\text-\mathsf{Type}}\xspace}
\newcommand{\typelepU}[1]{\ensuremath{{(#1)}\text-\mathsf{Type}_\UU}\xspace}
\let\ntype\typele
\let\ntypeU\typeleU
\let\ntypep\typelep
\let\ntypepU\typelepU
%\renewcommand{\set}{\ensuremath{\mathsf{Set}}\xspace}
%\newcommand{\setU}{\ensuremath{\mathsf{Set}_\UU}\xspace}
\newcommand{\prop}{\ensuremath{\mathsf{Prop}}\xspace}
\newcommand{\propU}{\ensuremath{\mathsf{Prop}_\UU}\xspace}
%Pointed types
\newcommand{\pointed}[1]{\ensuremath{#1_\bullet}}

%
%%% The empty type
\newcommand{\emptyt}{\ensuremath{\mathbf{0}}\xspace}

%%% The unit type
\newcommand{\unit}{\ensuremath{\mathbf{1}}\xspace}
\newcommand{\ttt}{\ensuremath{\star}\xspace}

%%% The two-element type
\newcommand{\bool}{\ensuremath{\mathbf{2}}\xspace}
\newcommand{\btrue}{{1_{\bool}}}
\newcommand{\bfalse}{{0_{\bool}}}
\newcommand{\belim}{{\elim_{\bool}}}

%%% Injections into binary sums and pushouts
\newcommand{\inlsym}{{\mathsf{inl}}}
\newcommand{\inrsym}{{\mathsf{inr}}}
\newcommand{\inl}{\ensuremath\inlsym\xspace}
\newcommand{\inr}{\ensuremath\inrsym\xspace}

%%% Natural numbers
\newcommand{\N}{\ensuremath{\mathbb{N}}\xspace}
%\newcommand{\N}{\textbf{N}}
\let\nat\N
\newcommand{\natp}{\ensuremath{\nat'}\xspace} % alternative nat in induction chapter

\newcommand{\zerop}{\ensuremath{0'}\xspace}   % alternative zero in induction chapter
\newcommand{\suc}{\mathsf{suc}}
\newcommand{\sucp}{\ensuremath{\suc'}\xspace} % alternative suc in induction chapter
\newcommand{\add}{\mathsf{add}}
\newcommand{\ack}{\mathsf{ack}}
\newcommand{\ite}{\mathsf{iter}}
\newcommand{\assoc}{\mathsf{assoc}}
\newcommand{\dbl}{\ensuremath{\mathsf{double}}}
\newcommand{\dblp}{\ensuremath{\dbl'}\xspace} % alternative double in induction chapter


% Join lists

\newcommand{\JList}[1]{\mathsf{JList}\,#1}
\newcommand{\List}[1]{\mathsf{List}\,#1}
\newcommand{\flatten}{\mathsf{flatten}}
\newcommand{\Jnil}{\mathsf{[]}}
\newcommand{\Jsing}[1]{\mathsf{[#1]}}
\newcommand{\Jconcat}[2]{#1 \mathop{+\!\!+} #2}

% Function definition with domain and codomain
\newcommand{\function}[4]{\left\{\begin{array}{rcl}#1 &
      \longrightarrow & #2 \\ #3 & \longmapsto & #4 \end{array}\right.}


%%% Sets
\newcommand{\bin}{\ensuremath{\mathrel{\widetilde{\in}}}}

%%% Macros for the formal type theory
\newcommand{\emptyctx}{\ensuremath{\cdot}}
\newcommand{\emptysub}{\ensuremath{\cdot}}
\newcommand{\ctx}{\ensuremath{\mathsf{ctx}}}
\newcommand{\idS}{\ensuremath{\mathsf{id}}} 
%\newcommand{\instSub}[1]{\ensuremath{(\idS, #1)}}
\newcommand{\instSub}[1]{\ensuremath{\langle\,#1\,\rangle}}

%\newcommand{\sub}[2]{[#1]\,{#2}}
\newcommand{\sub}[2]{{{#1}\,[#2]}}
\newcommand{\subTm}[2]{\sub{#1}{#2}_{\Tm}}
\newcommand{\subTy}[2]{\sub{#1}{#2}_{\Ty}}
\newcommand{\subTel}[2]{\sub{#1}{#2}_{\Tel}}
\newcommand{\subM}[2]{{#1}\,[#2]^M}
\newcommand{\subTmM}[2]{\subM{#1}{#2}_{\Tm}}
\newcommand{\subTyM}[2]{\subM{#1}{#2}_{\Ty}}
\newcommand{\subTelM}[2]{\subM{#1}{#2}_{\Tel}}

\newcommand{\cc}{\mathsf{c}}



\begin{document}

\maketitle

\begin{abstract}
  Type theory can be defined in a type theory as a quotient inductive-inductive type, but well-typed terms are littered with explicit coercions, i.e.\ \transp's, along equality constructors and proofs need to take account of coercions painstakingly alongside the interesting part.
  This mess has been dubbed the `transport hell' but typically suppressed in presentation for clarity, hiding the gap between the intention and the formalisation of type theory in type theory.
  In this paper, we aim for shortening the gap using quotient inductive-inductive-recursive types and definitions by overlapping patterns altogether, reducing the use of \transp's.
  As a case study, we investigate (parallel) substitution calculus and type theory with Coquand universes and $\Pi$-types as quotient inductive-inductive-recursive types and compare ours with quotient inductive-inductive definitions.
\end{abstract}

\section{Introduction} \label{sec:intro}
\LT{%
Points to elaborate:
\begin{enumerate}
  \item QIITs are littered with transports, causing too much pains while formalising a type theory.
  \item Rewrite rules are popular to make structural rules definitional but requires efforts to justify meta-theoretic properties in general.
  \item Further, the confluence check adopted by \Agda is based on complete development and requires one-step parallel reductions to have a confluent term, making a definition more complicated than necessary.
  \item Moreover, the use of rewrite rules is typically \emph{not} presented and discussed in literature. We are ignoring the elephant in the room.
  \item Instead, definitions by overlapping patterns (in theory) requires strong normalisation and local confluence, making definitions simpler to design.
  \item The rules for type substitutions are \emph{structural} shared with other type theories based on parallel substitution (in particular cwf).
  
\end{enumerate}
}


\paragraph*{Contributions}
\begin{itemize}
  \item Exploration of the use of quotient inductive-inductive-recursive types and definitions by overlapping patterns.
    In particular, we give in a type theory:
    \begin{itemize}
      \item a definition of parallel substitution calculus;
      \item a definition of type theory with Coquand universes and $\Pi$-types,
    \end{itemize}
   using quotient inductive-inductive-recursive types and definitions by overlapping patterns to define type substitution and term substitution partially.
  \item Comparison with other definitions using quotient inductive-inductive types.
\end{itemize}


\subsection{Plan of the paper}
\subsection{Related work}
\paragraph*{Formalisation of type theory in type theory}
\cite{Danielsson2006,Altenkirch2016a}
\cite{Altenkirch2017}
\cite{Munoz1998}
\cite{Dybjer1996,Castellan2021}
\LT{Dybjer \cite{Dybjer1996} already says cwf is inductive-recursive}

Substitution calculus is closely linked to the categorical model of type theory \emph{categories with families}~\cite{Dybjer1996} 
\paragraph*{Single substitution calculus}
\cite{Kaposi2023,Kaposi2024a}
\paragraph*{Schemata of inductive types}
\cite{Kaposi2019}

\section{Metatheory and formalisation}
\cite{UFP2013}
\LT{
Points to include:
\begin{enumerate}
  \item identity type $x =^{A} y$ for $x, y : A$, dependent identity type, $t =^{P}_{p} u \defeq \transfib{P}{p}{t} =^{P y} u$ for $t : P(x)$ and $u : P(y)$, heterogeneous equality~\cite{McBride1999} in \Agda $x \simeq y$ for $x : A$ and $y : B$
  \item We work with intensional type theory with uniqueness of identity proof and function extensionality (only used for NbE and the standard model for the extensional identity type).
  \item We use Agda to formally implement our definitions with the following options: \texttt{-{}-with-K}, \texttt{-{}-local-confluence-check}, \texttt{-{}-exact-split}, and \texttt{-{}-rewriting} using postulated equations~\cite{Licata2011} to introduce equality constructors for quotient types.
\end{enumerate}}

\subsection{Inductive-recursive types}
\cite{Dybjer2003,Dybjer2000,Dybjer1999}

\LT{Mention that the recursion part of induction-recursion are already not defined by elimination rule.}

\subsection{Definitions by overlapping patterns}
\cite{Cockx2014,Altenkirch2016a}
\subsection{Quotient inductive(-inductive)-recursive types}
\LT{Give an example of QIRT: join list?}
\subsection{Definitions by rewrite rules}
\cite{Cockx2020,Cockx2021}

Unfortunately, neither general schemata of quotient inductive-inductive-recursive types nor definitions by overlapping patterns have been developed or implemented in existing proof assistants.

\paragraph*{Local confluence}
\paragraph*{Strong normalisation}


\section{Type theories as quotient inductive-inductive-recursive types} \label{sec:QIIRTs}
\LT{
\begin{enumerate}
  \item Goal: no transports in the definition.
  \item Motivation: make type substitution definitional so that we do not have to apply transport along structural rules for types explicitly.
  \item Develop the quotient inductive-inductive-recursive definition of type theory step by step (which needs to be locally confluent and terminating).
    \begin{enumerate}
      \item parallel substitution: type substitution
      \item the type of elements of $A$: term substitution needs to be split into two definitional substitution and explicit term substitution to maintain local confluence.
      \item $\Pi$-type: because of the substitution lifting $\sigma^+ \defeq (\pi_1\idS, \pi_2\idS)$ is used we need to introduce a definitional lifting to maintain local confluence.
        We prefer categorical combinator and avoid using $\left< t \right>$ in the definition, which would introduce transports in the definition.
      \item Other type formers can be introduced as usual (as long as we use categorical combinators).
    \end{enumerate}
\end{enumerate}
}
\subsection{Parallel substitution calculus} \label{subsec:SC}
We begin with recalling the definition of (parallel) substitution calculus~\cite{Martin-Lof1992} as a quotient inductive-inductive-type, emphasising the use of transports in its definition, which will be developed into a more complicated type theory.
This calculus is a fragment of type theory studied by Altenkirch and Kaposi~\cite{Altenkirch2016a}, which can be defined using the general schema~\cite{Kaposi2019} introduced by Kaposi\ et~al.

Substitution calculus has a type $\Ctx$ of contexts, a type $\Ty\,\Gamma$ of types under some context $\Gamma : \Ctx$, a type $\Sub\;\Gamma\;\Delta$ of substitutions from the domain $\Gamma$ and the codomain $\Delta$, and a type $\Tm\;\Gamma\;A$ of terms under a context $\Gamma$ and its type $A$. 
These types amount to the following types indexed by types being defined (hence inductive-inductive):
\begin{alignat*}{3}
  \Ctx   & : \Set                   \\
  \Ty    & : \Ctx \to \Set          \\
  \Sub   & : \Ctx \to \Ctx \to \Set \\
  \Tm    & : (\Gamma : \Ctx) \to \Ty\,\Gamma \to \Set
\end{alignat*}
The type $\Ctx$ has two constructors $\emptyctx$ for the empty context and $\blank,\blank$ for context extension:
\begin{alignat*}{3}
  \emptyctx & : \Ctx \\
  \blank,\blank & : (\Gamma : \Ctx) \to \Ty\,\Gamma \to \Ctx
\end{alignat*}
where the context extension $\blank,\blank$ requires that the type $\Ty$ indexed by a context $\Gamma$, underscoring the nature of inductive-inductive definition.

Type substitution takes a substitution $\sigma : \Sub\,\Gamma\,\Delta$ and a type under $\Delta$ to form a type under $\Gamma$, so type substitution as a constructor has the type
\begin{alignat*}{3}
  [\blank]\blank &: \implicit{\Gamma, \Delta} \; \Sub\,\Gamma\,\Delta \to \Ty\,\Delta \to \Ty\,\Gamma,
\end{alignat*}
If we were defining cwfs, it would be enough to require type substitution for $\Ty$.
However, we need a ground type $\UU$ in $\Ty$ for any context $\Gamma$ for our inductive definition (i.e.\ the initial cwf), 
otherwise the type $\Ty$ would be empty as well as other types:
\begin{alignat*}{3}
  \UU & : \implicit{\Gamma} & \Ty\, \Gamma.
\end{alignat*}

Note that $\Gamma$ and $\Delta$ (coloured {\color{gray}grey}) are quantified implicitly above.
For now, $\UU$ serves as a constant in $\Ty$, but we will reuse $\UU$ for the type of small types later.

Substitutions from $\Gamma$ to $\Delta$ can be understood intuitively as lists of terms of type $A$ under the context $\Gamma$ for each $A$ in $\Delta$, so we have the empty substitution $\emptysub$ and substitutions $\sigma, t$ extended by some term~$t$:
\begin{alignat*}{3}
  \emptysub & : \implicit{\Gamma}\;\Sub\,\Gamma\,\emptyctx \\
  \blank,\blank & : \implicit{\Gamma, \Delta, A}\;(\sigma : \Sub\,\Gamma\,\Delta) \to \Tm\,\Gamma\,([ \sigma ]\, A) \to \Sub\,\Gamma,(\Delta, A),
\end{alignat*}
Note that type substitution $[\sigma]\,A$ is needed, because $A$ is well-formed under the context $\Delta$ instead of $\Gamma$.
As substitution calculus is closely related to cwfs (more precisely, it is the initial model), substitutions also have the identity substitution $\idS$ and composition $\blank;\blank$
\begin{alignat*}{3}
  \idS & : \implicit{\Gamma}\;\Sub\,\Gamma\,\Gamma \\
  \blank;\blank & : \implicit{\Gamma, \Delta, \Theta}\;\Sub\,\Gamma\,\Delta \to \Sub\,\Delta\,\Theta \to \Sub\,\Gamma\,\Theta \\
\end{alignat*}
satisfying certain laws introduced later.
Context comprehension is given by projections from \emph{non-empty} substitutions $\Sub\;\Gamma\;(\Delta, A)$:
\begin{alignat*}{3}
  \pi_1 & : \implicit{\Gamma, \Delta, A}\;\Sub\,\Gamma\,(\Delta, A) \to \Sub\,\Gamma\,\Delta \\
  \pi_2 & : \implicit{\Gamma, \Delta, A}\;(\sigma : \Sub\,\Gamma\,(\Delta, A)) \to \Tm\,\Gamma\,([ \pi_1\,\sigma ]\, A)
\end{alignat*}
where $\pi_1\,\sigma$ and $\pi_2\,\sigma$ can be intuitively understood as the \emph{tail} and the \emph{head} of a non-empty substitution $\sigma : \Sub\;\Gamma\;(\Delta, A)$ respectively.
Again, a type substitution $[\pi_1\,\sigma]\,A$ is needed for $\pi_2$, because $A$ is under the context $\Delta$ instead of $\Gamma$.
Finally, we also have term substitution $[\sigma]\,t$ whose type is $[\sigma]\,A$ for a term $t : \Tm\;\Delta\;A$
\begin{alignat*}{3}
  [\blank] \blank & : \implicit{\Gamma,\Delta, A}\;(\sigma : \Sub\,\Gamma\,\Delta) \to \Tm\,\Delta\,A \to \Tm\,\Gamma\,([\sigma]\, A)
\end{alignat*}
where the symbol $[\blank]\blank$ is overloaded.

We also have equality constructors to stipulate properties for type substitution:
\begin{alignat*}{3}
  [\idS]_T & : \implicit{\Gamma, A}                               && [ \idS ] \,A         && =^{\Ty\,\Gamma} A \\
  [;]_T    & : \implicit{\Gamma, \Delta, \Theta, \sigma, \tau, A} && [ \sigma ; \tau ]\,A && =^{\Ty\,\Gamma} [ \sigma ]\;([ \tau ]\;A) \\
  []\UU      & : \implicit{\Gamma, \Delta, \sigma}                  && [ \sigma ]\,\UU        && =^{\Ty\,\Gamma} \UU
\end{alignat*}

Laws for substitutions are stipulated using equality constructors.
\begin{alignat*}{5}
  \mathsf{idr}    & : \implicit{\Gamma, \Delta, \sigma} && {\sigma ; \idS_{\Delta}} && =^{\Sub\,\Gamma\,\Delta} && {\sigma} \\
  \mathsf{idl}    & : \implicit{\Gamma, \Delta, \sigma} && {\idS_{\Gamma} ; \sigma} && =^{\Sub\,\Gamma\,\Delta} && {\sigma} \\
  ;\text{-}\mathsf{assoc} & : \implicit{\Gamma, \Delta, \Theta, \Xi, \sigma, \tau, \gamma} && (\sigma ; \tau) ; \gamma && =^{\Sub\,\Gamma\,\Theta} &&  \sigma ; (\tau ; \gamma) \\
  \mathsf{concat} & : \implicit{\Gamma, \Delta, \Theta, \sigma, \tau, A, t} && \sigma ; (\tau , t)      && =^{\Sub\,\Gamma\,(\Theta, A)} &&  (\sigma ; \tau) , \alert{\transfib{\Tm\,\Gamma}{[;]_{\Ty}^{-1}}{\color{black}[ \sigma ] t}} \\
  \emptyctx\eta   & : \implicit{\Gamma, A, \sigma} && \sigma                   && =^{\Sub\,\Gamma\,\emptyctx} & \emptysub \\
  \pi\eta         & : \implicit{\Gamma, \Delta, \sigma} && \sigma                   && =^{\Sub\,\Gamma\,(\Delta, A)} &&  (\pi_1 \sigma, \pi_2 \sigma) \\
\end{alignat*}
In particular, $\blank;\blank$ acts like a list concatenation, so conceptually $\sigma; (\tau, t)$ is equal to $(\sigma; \tau), t$ whereas the term $t$ is of type $\Tm\;\Delta\;([\tau]\;A)$. 
Moving the term $t$ from the left-hand side to the right-hand side requires us to apply substitution $\sigma$ to the term $t$ to obtain an inhabitant of $\Tm\;\Gamma\;[\sigma][\tau]\,A$.
By definition of the substitution extension, however, $[\sigma]\;t$ requires to be $\Tm\;\Gamma\;[\sigma; \tau]\,A$ instead.
Therefore, we need to transport $[\sigma]\;t$ explicitly along $[;]_{\Ty}^{-1}\colon [\sigma]\,[\tau]\;A = [\sigma;\tau]\,A$ to obtain a term of type $\Tm\;\Gamma\;([\sigma;\tau]\;A)$.

The tail and the head of a non-empty substitution $(\sigma, t)$ is obviously $\sigma$ and $t$ respectively:
\begin{alignat*}{5}
  \pi_1\beta      & : \implicit{\Gamma, \Delta, \Theta, \sigma, A, t} && \pi_1(\sigma , t)        && =^{\Sub\,\Gamma,\Delta} &&  \sigma \\
  \pi_2\beta      & : \implicit{\Gamma, \Delta, \Theta, \sigma, A, t} && \pi_2(\sigma , t)        && =^{\Tm\,\Gamma\,([\blank]\,A)}_{\alert{\pi_1\beta}} &&  t, 
\end{alignat*}
whereas $\pi_2(\sigma, t)$ is an habitant of $\Tm\;\Gamma\;([\pi_1\,(\sigma, t)]\,A)$ instead of $\Tm\;\Gamma\;([\sigma]\,A)$ on the right-hand side. 
Therefore, we have to transport $\pi_2(\sigma, t)$ along $\pi_1\beta$, so the above equality constructor is, in fact, $\transfib{\Tm\;\Gamma\;([\blank]\,A)}{\pi_1\beta}{\pi_2(\sigma, t)} =^{\Tm\;\Gamma\;([\sigma]\,A)} t$.

Similarly, for term substitution, terms are transported along the corresponding rules:
\begin{alignat*}{5}
  [\idS]_t         & : \implicit{\Gamma, A, t} && {[\,\idS\,]\,t}         && =^{\Tm\,\Gamma}_{\alert{[\idS]_\Ty}}  && t \\
  [;]_t            & : \implicit{\Gamma, \Delta, \Theta, \sigma, \tau, t} && {[\,\sigma ; \tau\,]\,t} && =^{\Tm\,\Gamma}_{\alert{[;]_{\Ty}}}   && {[ \sigma ]\,[ \tau ]\,t} \\
\end{alignat*}
equivalent to the following (homogeneous) identities
\[
  \transfib{\Tm\;\Gamma}{[\idS]_{\Ty}}{[\idS]\,t} =^{\Tm\,\Gamma\,A} t
  \quad\text{and}\quad
  \transfib{\Tm\;\Gamma}{[;]_{\Ty}}{[\sigma;\tau]\,t} =^{\Tm\,\Gamma\,([\sigma]\,[\tau]A)} [\sigma]\,[\tau]\,t
\]
respectively.
Constructors introduced so far complete the definition of substitution calculus.

The use of transports in the formal definition fixes type mismatches but hinders equational reasoning about these terms, even for simple facts.
\begin{example}
  Given substitutions $\sigma : \Sub\;\Gamma\; \Delta$ and $\tau : \Sub\;\Delta\;(\Theta, A)$ for any $A : \Ty\,\Theta$, we may apply the projection $\pi_2$ to the composite $(\sigma; \tau)$ to access the first term $\pi_2(\sigma; \tau)$ of type $[\sigma;\tau] A$ under the context $\Gamma$, and this term should be equal to the first term $\pi_2\,\tau$ of $\tau$ after applying the substitution $\sigma$. 
  In short, the following equality apparently holds
  \[
    \pi_2\,(\sigma ; \tau) = [\sigma] (\pi_2\,\tau)
  \]
  by a back-of-the-envelope calculation
  \begin{equation} \label{eq:pi2-comp-proof}
    \pi_2\,(\sigma ; \tau) 
    = \pi_2\,(\sigma; (\pi_1\,\tau, \pi_2\,\tau))
    = \pi_2\,(\sigma;\pi_1\,\tau, [\sigma]\,(\pi_2\,\tau))
    = [\sigma] (\pi_2\,\tau).
  \end{equation}
  Yet, the left-hand side is a term of type $[\pi_1\,(\sigma;\tau)] A$, but the other is $[\sigma] [\pi_1\,\tau] A$.
  Hence the above identity does not even make sense, since their types do not match.
  Alas, instead, we have to write $\pi_2\,(\sigma ; \tau) =^{\Tm\,\Gamma}_{p} [\sigma] (\pi_2\,\tau)$ or, equivalently
  \begin{equation}\label{eq:pi2-comp-real-proof}
    ([\pi_1(\sigma; \tau)]\,A, \pi_2(\sigma; \tau)) =^{(A : \Ty\,\Gamma) \times (\Tm \Gamma A)} ([\sigma]\,[\pi_1\,\tau]A, [\sigma] (\pi_2\,\tau))
  \end{equation}
  as inhabitants of a $\Sigma$-type, so we can reason about term equalities along with type equalities.
  Moreover, in~\eqref{eq:pi2-comp-proof} we have used the rule $\mathsf{concat}$ which introduced another transported term, so we will have to eliminate that $\transp$ to derive the right hand side.

  To better illustrate the annoyance, note that a complete proof of \eqref{eq:pi2-comp-real-proof} requires us to show each of following equations:
  \begin{alignat*}{3}
         & ([\pi_1(\sigma; \tau)]\,A                    &&, \pi_2\,(\sigma ; \tau)) \label{eq:pi2-proof-1} \\
    = {} & ([\pi_1(\sigma; (\pi_1\tau , \pi_2\tau))]\,A &&, \pi_2\,(\sigma; (\pi_1\,\tau, \pi_2\,\tau))) \\
    = {} & ([\pi_1(\sigma;\pi_1\,\tau, t)]\,A &&, \pi_2\,(\sigma;\pi_1\,\tau, t)) \\
    = {} & ([\pi_1(\sigma;\pi_1\,\tau, t)]\,A &&, \pi_2\,(\sigma;\pi_1\,\tau, t)) \nonumber \\
    = {} & (\sub{\sigma;\pi_1\tau}{A} &&, [\sigma] (\pi_2\,\tau)) \nonumber
  \end{alignat*}
  where $p \judgeq [;]_{\Ty}^{-1} : \sub{\sigma}{\sub{\pi_1\,\tau}{A}} \to \sub{\sigma; \pi_1 \tau}{A}$ and $t \judgeq\transfib{\Tm\,\Gamma}{p}{[\sigma]\,(\pi_2\,\tau)}$,
explicitly.
The first three equations and the last equation follow from Lemma~2.3.4 and 2.3.2 in \cite{UFP2013} respectively, which cannot be proved naively as we did in \eqref{eq:pi2-comp-proof}.
\end{example}

To retain the intuitive way of reasoning such as \eqref{eq:pi2-comp-proof} \emph{formally}, we make type substitution \emph{definitional}, since the root cause is the type mismatch after type substitution.

One possibility is to define substitution calculus as a quotient inductive-inductive-\emph{recursive} type whereas type substitution is a recursion.
Considering that $[ \idS ]_{\Ty}\,A \jdeq A$ and $[ \sigma ; \tau ]_{\Ty}\,A \jdeq [ \sigma ]_{\Ty}\;([ \tau ]_{\Ty}\;A)$ along do not make $[\blank]_{\Ty}\blank$ a function, we may attempt to define it by:
\begin{alignat*}{3}
[\blank]_{\Ty} \blank &: \implicit{\Gamma, \Delta} \; \Sub\,\Gamma\,\Delta \to \Ty\,\Delta \to \Ty\,\Gamma \\
[ \sigma ]_{\Ty}\,\UU           & = \UU
\end{alignat*}
Now, $[\idS]_{\Ty}\,A$ and $[\sigma; \tau]_{\Ty}\,A$ can be proved equal to $A$ and $[\sigma]_{\Ty}\,[\tau]_{\Ty}\,A$ respectively.
Still, they are not definitional but propositional, so this recursion does not eliminate any transport in the definition.

Our second attempt is to define type substitution with additional rules:
\begin{alignat*}{3}
[\blank]_{\Ty} \blank &: \implicit{\Gamma, \Delta} \; \Sub\,\Gamma\,\Delta \to \Ty\,\Delta \to \Ty\,\Gamma \\
[ \idS ]_{\Ty}\,A             & = A \\
[ \sigma ; \tau ]_{\Ty}\,A    & = [ \sigma ]_{\Ty}\;([ \tau ]_{\Ty}\;A)
[ \sigma ]_{\Ty}\,\UU         & = \UU \\
\end{alignat*}
However, this definition will \emph{not} compute $[\sigma]_{\Ty}\,\UU$ in most type theories.
\LT{briefly explain how this definition should be translated to an eliminator}


\begin{alignat*}{3}
  [\blank]_{\Ty} \blank &: \implicit{\Gamma, \Delta} \; \Sub\,\Gamma\,\Delta \to \Ty\,\Delta \to \Ty\,\Gamma \\
[ \idS ]_{\Ty}\,A             & \jdeq A \\
[ \sigma ; \tau ]_{\Ty}\,A    & \jdeq [ \sigma ]_{\Ty}\;([ \tau ]_{\Ty}\;A) \\
[ \pi_1(\sigma, t) ]_{\Ty}\,A & \jdeq [\sigma]_\Ty\,A \\
[ \pi_1(\sigma; \tau) ]_{\Ty}\,A & \jdeq [\sigma]_\Ty ([\pi_1\tau]_\Ty \,A) \\
[ \sigma ]_{\Ty}\,\UU           & \jdeq \UU
\end{alignat*}
With this type substitution being definitional for each clause, we update other equality constructors without any $\transp$:
\begin{alignat*}{5}
  \mathsf{concat} & : \implicit{\Gamma, \Delta, \Theta, \sigma, \tau, A, t} &&\sigma ; (\tau , t)      && =^{\Sub\,\Gamma\,(\Theta, A)} &&  (\sigma ; \tau) , [ \sigma ] t \\
  \pi_2\beta      & : \implicit{\Gamma, \Delta, \Theta, \sigma, A, t} && \pi_2(\sigma , t)        && =^{\Tm\,\Gamma\, A} &&  t \\
  [\idS]t         & : \implicit{\Gamma, A, t} && {[\,\idS\,]\,t}          && =^{\Tm\,\Gamma\,A} && t \\
  [;]t            & :\implicit{\Gamma, \Delta, \Theta, \sigma, \tau, t} && {[\,\sigma ; \tau\,]\,t} && =^{\Tm\,\Gamma\,[\sigma ; \tau] A} && {[ \sigma ]\,[ \tau ]\,t} \\
\end{alignat*}

\LT{Revisit our previous example.}
\begin{proposition}[Local confluence]
  Type substitution $[\blank]_{\Ty}\blank$ is locally confluent.
  \danger
\end{proposition}
\begin{proposition}[Coherence]
  $[\sigma]_{\Ty}\,A$ is (propositionally) equal to $[\tau]_{\Ty}\,A$ whenever $\sigma = \tau$.
  \danger
\end{proposition}

\subsection{... with a Coquand universe} \label{subsec:SC+U}
\cite{Coquand2013}
\begin{alignat*}{3}
  \UU  & : \implicit{\Gamma}\;\Ty\, \Gamma \\
  \El & : \implicit{\Gamma}\;\Tm\,\Gamma\;\UU \to \Ty\,\Gamma \\
  [ \sigma ]_{\Ty}\,(\El\, u) & \jdeq \El\,([\sigma]_{\Tm}{u}) \\
\end{alignat*}
\begin{alignat*}{3}
  [\blank]_{\Tm}\blank & : (\sigma : \Sub\,\Gamma\,\Delta) \to \Tm\,\Delta\,A \to \Tm\,\Gamma\,([\sigma]_{\Ty}\,A) \\
[ \idS ]_{\Tm}\,t          & \jdeq t \\
[ \sigma ; \tau ]_{\Tm}\,t & \jdeq [ \sigma ]_{\Tm}\;([ \tau ]_{\Tm}\;t) \\
[ \pi_1(\sigma, t) ]_{\Tm}\,t & \jdeq [\sigma]_\Tm\,t \\
[ \pi_1(\sigma; \tau) ]_{\Tm}\,t & \jdeq [\sigma]_\Tm\; ([\pi_1\tau]_\Tm \,t) \\
[ \sigma ]_{\Tm}\,t        & \jdeq [ \sigma ]\,t, \quad \text{otherwise}  \\
\end{alignat*}

\begin{proposition}[Local confluence]
  Type and term substitutions $[\blank]_{\Ty}\blank$ and $[\blank]_{\Tm}\blank$ are locally confluent.
  \danger
\end{proposition}

\begin{proposition}[Coherence]
  The following statements hold:
  \danger
  \begin{enumerate}
    \item $[\sigma]_{\Ty}\,A = [\tau]_{\Ty}\,A$ whenever $\sigma = \tau$.
    \item $[\sigma]_{\Tm}\,A = [\tau]_{\Tm}\,A$ whenever $\sigma = \tau$.
  \end{enumerate}
\end{proposition}
\subsection{... and \texorpdfstring{$\Pi$}{Π}-types} \label{subsec:SC+U+Pi}
\begin{alignat*}{3}
  \Pi     &: \implicit{\Gamma}            && (A : \Ty\,\Gamma) \to \Ty\,(\Gamma, A) \to \Ty\,\Gamma \\
  \lambda &: \implicit{\Gamma, A, B}      && \Tm\,(\Gamma, A)\,B \to \Tm\,\Gamma\,(\Pi\,A\,B) \\
  \mathsf{app} &: \implicit{\Gamma, A, B} && \Tm\,\Gamma\,(\Pi\,A\,B) \to \Tm\,(\Gamma, A)\,B \\
\end{alignat*}
\begin{alignat*}{3}
  \blank^+ & : \implicit{\Gamma, \Delta, A}\; (\sigma : \Sub\,\Gamma,\Delta) \to (A : \Ty\,\Delta) \to \Sub\,(\Gamma, [\sigma]_{\Ty}\,A)\, (\Delta, A) \\
  \sigma^+ & \defeq (\pi_1\idS ; \sigma, \pi_2\idS) \\
\end{alignat*}
\begin{alignat*}{3}
  \blank\uparrow \blank & : (\sigma : \Sub\,\Gamma,\Delta) \to (A : \Ty\,\Delta) \to \Sub\,(\Gamma, [\sigma]_{\Ty}\,A)\, (\Delta, A) \\
\idS                \uparrow A & \jdeq \idS \\
\sigma ; \tau       \uparrow A & \jdeq (\sigma \uparrow \sub{\tau}{A} ) ; (\tau \uparrow A) \\
\pi_1(\sigma, t)    \uparrow A & \jdeq \sigma \uparrow A \\
\pi_1(\sigma; \tau) \uparrow A & \jdeq \sigma \uparrow (\sub{\pi_1 \tau}{A}) ; (\pi_1 \tau \uparrow A) \\
\sigma              \uparrow A & \jdeq \sigma^+ , \quad \text{otherwiese} \\
  [ \sigma ]_{\Ty}\,(\Pi\,A\,B) & \jdeq \Pi\,(\sub{\sigma}{A})\,(\sub{\sigma\uparrow A}{B})
\end{alignat*}

\begin{alignat*}{3}
  \mathsf{[]\lambda} & : [\sigma]_{\Tm} (\lambda\,t) && =^{\Tm\,\Gamma\,([\sigma]_{\Ty}(\Pi\,A\,B))} && \lambda\,([\sigma \uparrow A]_{\Tm}\,t) \\
  \Pi\beta & : \mathsf{app}(\lambda\,t) && =^{\Tm\,\Gamma,A} {} && t \\
  \Pi\eta & :  t && =^{\Tm\,\Gamma,A} {} && \lambda(\mathsf{app}\,t)
\end{alignat*}

%% ⟨_⟩ : Tm Γ A → Sub Γ (Γ , A)
%% ⟨ t ⟩ = idS , t
%% 
%% _$$_
%%   : (t : Tm Γ (Π A B)) (u : Tm Γ A)
%%   → Tm Γ ([ ⟨ u ⟩ ] B)
%% t $$ u = [ ⟨ u ⟩ ]t app t
\begin{alignat*}{3}
  \left<\blank\right> & : \Tm\,\Gamma\,A \to \Sub\,\Gamma\,(\Gamma, A) \\
  \left< t \right> & \defeq (\idS , t) \\
  \blank \$\blank & : \Tm\,\Gamma\,(\Pi\,A\,B) \to (u : \Tm\,\Gamma\,A) \to \Tm\,\Gamma\,([ \left< u \right> ]\,B) \\
  t \mathop{\$} u & \defeq [ \left< u \right> ]_{\Tm} (\mathsf{app}\,t)
\end{alignat*}

\begin{proposition}
  For every substitution $\sigma : \Sub\,\Gamma\,\Delta$, type $A : \Ty\,\Delta$, and term $t : \Tm\,\Delta\,A$, the following statements hold:
  \danger
  \[
    \sigma \uparrow A = (\pi_1 \idS ; \sigma , \pi_2 \idS)
  \]
  and
  \[
    [ \sigma ]_{\Tm}\,t = [ \sigma ]\,t
  \]
\end{proposition}

\begin{proposition}
  The following statements hold:
  \danger
  \begin{enumerate}
    \item $\sigma \uparrow A = \tau \uparrow A$ whenever $\sigma = \tau$.
    \item $[\sigma]_{\Ty}\,A = [\tau]_{\Ty}\,A$ whenever $\sigma = \tau$.
    \item $[\sigma]_{\Tm}\,t = [\tau]_{\Tm}\,t$ whenever $\sigma = \tau$.
  \end{enumerate}
\end{proposition}
\subsection{... and other type formers} \label{subsec:SC+U+Pi+more}
\LT{%
\begin{enumerate}
  \item Discuss the unit type.
  \item Discuss the Boolean type.
  \item Discuss the type of natural numbers.
  \item Discuss the extensional identity type.
\end{enumerate}
}

\section{The eliminator and the standard model}
\LT{%
\begin{enumerate}
  \item Introduce motives and methods for the definition in \Cref{subsec:SC+U+Pi}.
  \item The rule $\mathsf{Elim}_{\Ty}\,([ \sigma ]_{\Ty}\,A) = [ \mathsf{Elim}_{\Sub}\,\sigma ]^{A}_{\Ty}(\mathsf{Elim}_{\Ty}\,A)$ is not defined but provable. 
  \item $\mathsf{Elim}_{\Tm}\,([ \sigma ]_{\Tm}\,t) = [ \mathsf{Elim}_{\Sub}\,\sigma ]^{A}_{\Tm}(\mathsf{Elim}_{\Tm}\,t)$ is proved.
\end{enumerate}
}

\section{Equivalence to the quotient inductive-inductive definition of type theory}
\LT{%
\begin{enumerate}
  \item Show that two definitions are equivalent (ongoing).
  \item Show that methods are equivalent (ongoing).
\end{enumerate}
}

\section{Future work}

\paragraph*{Type theory in (observational) type theory}


\paragraph*{A general schema of quotient inductive-inductive-recursive types}
\subparagraph*{Formalising definitions by overlapping patterns}

\IfFileExists{./reference.bib}{\bibliography{reference}}{\bibliography{ref}}

\appendix

\section{Complete definitions}
\begin{alignat*}{3}
  \Ctx      & : && \Set                   \\
  \Ty       & : && \Ctx \to \Set          \\
  \Sub      & : && \Ctx \to \Ctx \to \Set \\
  \Tm       & : && (\Gamma : \Ctx) \to \Ty\,\Gamma \to \Set \\
  \emptyctx & : && \Ctx \\
  \blank,\blank & : && (\Gamma : \Ctx) \to \Ty\,\Gamma \to \Ctx \\
  [\blank]\blank & : \implicit{\Gamma, \Delta} && \Sub\,\Gamma\,\Delta \to \Ty\,\Delta \to \Ty\,\Gamma \\
  \emptysub & : \implicit{\Gamma} && \Sub\,\Gamma\,\emptyctx \\
  \blank,\blank & : \implicit{\Gamma, \Delta, A} && (\sigma : \Sub\,\Gamma\,\Delta) \to \Tm\,\Gamma\,([ \sigma ]_{\Ty} A) \to \Sub\,\Gamma,(\Delta, A) \\
  \idS & : \implicit{\Gamma} && \Sub\,\Gamma\,\Gamma \\
  \blank;\blank & : \implicit{\Gamma, \Delta, \Theta} && \Sub\,\Gamma\,\Delta \to \Sub\,\Delta\,\Theta \to \Sub\,\Gamma\,\Theta \\
  \pi_1 & : \implicit{\Gamma, \Delta, A} && \Sub\,\Gamma\,(\Delta, A) \to \Sub\,\Gamma\,\Delta \\
  \pi_2 & : \implicit{\Gamma, \Delta, A} && (\sigma : \Sub\,\Gamma\,(\Delta, A)) \to \Tm\,\Gamma\,([ \pi_1\,\sigma ]\, A) \\
  [\blank] \blank & : \implicit{\Gamma,\Delta, A} && (\sigma : \Sub\,\Gamma\,\Delta) \to \Tm\,\Delta\,A \to \Tm\,\Gamma\,([\sigma]\, A) \\
  \UU     & : \implicit{\Gamma} && \Ty\, \Gamma \\
  \El     & : \implicit{\Gamma} && \Tm\,\Gamma\,U \to \Ty\,\Gamma \\
  \Pi     & : \implicit{\Gamma} && (A : \Ty\,\Gamma) \to \Ty\,(\Gamma, A) \to \Ty\,\Gamma \\
\end{alignat*}

\begin{alignat*}{5}
[ \idS ]_{\Ty}\,A             & \jdeq A \\
[ \sigma ; \tau ]_{\Ty}\,A    & \jdeq [ \sigma ]_{\Ty}\;([ \tau ]_{\Ty}\;A) \\
[ \pi_1(\sigma, t) ]_{\Ty}\,A & \jdeq [\sigma]_\Ty\,A \\
[ \pi_1(\sigma; \tau) ]_{\Ty}\,A & \jdeq [\sigma]_\Ty ([\pi_1\tau]_\Ty \,A) \\
[ \sigma ]_{\Ty}\,\UU           & \jdeq \UU \\
[ \sigma ]_{\Ty}\,(\El\, u) & \jdeq \El\,([\sigma]_{\Tm}{u}) \\
[ \sigma ]_{\Ty}\,(\Pi\,A\,B) & \jdeq \Pi\,(\sub{\sigma}{A})\,(\sub{\sigma\uparrow A}{B}) \\
\idS                \uparrow A & \jdeq \idS \\
\sigma ; \tau       \uparrow A & \jdeq (\sigma \uparrow \sub{\tau}{A} ) ; (\tau \uparrow A) \\
\pi_1(\sigma, t)    \uparrow A & \jdeq \sigma \uparrow A \\
\pi_1(\sigma; \tau) \uparrow A & \jdeq \sigma \uparrow (\sub{\pi_1 \tau}{A}) ; (\pi_1 \tau \uparrow A) \\
\sigma              \uparrow A & \jdeq (\pi_1 \idS ; \sigma , \pi_2 \idS) & \text{otherwiese} \\
 [ \idS ]_{\Tm}\,t                & \jdeq t \\
 [ \sigma ; \tau ]_{\Tm}\,t       & \jdeq [ \sigma ]_{\Tm}\;([ \tau ]_{\Tm}\;t) \\
 [ \pi_1(\sigma, t) ]_{\Tm}\,t    & \jdeq [\sigma]_\Tm\,t \\
 [ \pi_1(\sigma; \tau) ]_{\Tm}\,t & \jdeq [\sigma]_\Tm\; ([\pi_1\tau]_\Tm \,t) \\
 [ \sigma ]_{\Tm}\,t              & \jdeq [ \sigma ]\,t & \text{otherwise}
\end{alignat*}

\begin{alignat*}{5}
  \mathsf{idr}    & : \implicit{\Gamma, \Delta, \sigma} && {\sigma ; \idS_{\Delta}} && =^{\Sub\,\Gamma\,\Delta} && {\sigma} \\
  \mathsf{idl}    & : \implicit{\Gamma, \Delta, \sigma} && {\idS_{\Gamma} ; \sigma} && =^{\Sub\,\Gamma\,\Delta} && {\sigma} \\
  ;\text{-}\mathsf{assoc} & : \implicit{\Gamma, \Delta, \Theta, \sigma, \tau, \gamma} && (\sigma ; \tau) ; \gamma && =^{\Sub\,\Gamma\,\Theta} &&  \sigma ; (\tau ; \gamma) \\
  \mathsf{concat} & : \implicit{\Gamma, \Delta, \Theta, \sigma, \tau, A, t} &&\sigma ; (\tau , t)      && =^{\Sub\,\Gamma\,(\Theta, A)} &&  (\sigma ; \tau) , [ \sigma ] t \\
  \emptyctx\eta   & : \implicit{\Gamma, \sigma} && \sigma                   && =^{\Sub\,\Gamma\,\emptyctx} & \emptysub \\
  \pi\eta         & : \implicit{\Gamma, A, \sigma} && \sigma                   && =^{\Sub\,\Gamma\,(\Delta, A)} &&  (\pi_1 \sigma, \pi_2 \sigma) \\
  \pi_1\beta      & : \implicit{\Gamma, \Delta, \Theta, \sigma, A, t} && \pi_1(\sigma , t)        && =^{\Sub\,\Gamma,\Delta} &&  \sigma \\
  \pi_2\beta      & : \implicit{\Gamma, \Delta, \Theta, \sigma, A, t} && \pi_2(\sigma , t)        && =^{\Tm\,\Gamma\, A} &&  t \\
  [\idS]t         & : \implicit{\Gamma, A, t} && {[\,\idS\,]\,t}          && =^{\Tm\,\Gamma\,A} && t \\
  [;]t            & :\implicit{\Gamma, \Delta, \Theta, \sigma, \tau, t} && {[\,\sigma ; \tau\,]\,t} && =^{\Tm\,\Gamma\,[\sigma ; \tau] A} && {[ \sigma ]\,[ \tau ]\,t} \\
\end{alignat*}

\section{Formal definitions in \Agda}
\end{document}
