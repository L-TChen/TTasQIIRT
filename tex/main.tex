
\documentclass[a4paper,UKenglish,numberwithinsect,cleveref,thm-restate]{lipics-v2021}
%This is a template for producing LIPIcs articles. 
%See lipics-v2021-authors-guidelines.pdf for further information.
%for A4 paper format use option "a4paper", for US-letter use option "letterpaper"
%for british hyphenation rules use option "UKenglish", for american hyphenation rules use option "USenglish"
%for section-numbered lemmas etc., use "numberwithinsect"
%for enabling cleveref support, use "cleveref"
%for enabling autoref support, use "autoref"
%for anonymousing the authors (e.g. for double-blind review), add "anonymous"
%for enabling thm-restate support, use "thm-restate"

%\pdfoutput=1 %uncomment to ensure pdflatex processing (mandatatory e.g. to submit to arXiv)
%\hideLIPIcs  %uncomment to remove references to LIPIcs series (logo, DOI, ...), e.g. when preparing a pre-final version to be uploaded to arXiv or another public repository

%\graphicspath{{./graphics/}}%helpful if your graphic files are in another directory

\bibliographystyle{plainurl}

\title{Type theories as quotient inductive-recursive types}
\author{Liang-Ting Chen\footnote{Corresponding author; authors are listed in alphabetical order.}}{Institute of Information Science, Academia Sinica, Taiwan \and \url{http://l-tchen.github.io}}{ltchen@iis.sinica.edu.tw}{https://orcid.org/0000-0002-3250-1331}{Supported by the National Science and Technology Council of Taiwan under grant NSTC [funding].}
\author{Tzu-Chun Tsai}{Institute of Information Science, Academia Sinica, Taiwan}{gene0905@icloud.com}{}{Supported by the National Science and Technology Council of Taiwan under grant NSTC 112-2221-E-001-003-MY3.}
\authorrunning{L.-T.~Chen and T.-C.~Tsai}
\Copyright{Liang-Ting Chen and Tzu-Chun Tsai}
\ccsdesc[500]{Theory of computation~Type theory}
\keywords{inductive-inductive types, quotient inductive types, inductive-recursive types, substitution calculus, category with families}
\relatedversion{} %optional, e.g. full version hosted on arXiv, HAL, or other respository/website
%\relatedversiondetails[linktext={opt. text shown instead of the URL}, cite=DBLP:books/mk/GrayR93]{Classification (e.g. Full Version, Extended Version, Previous Version}{URL to related version} %linktext and cite are optional

\supplement{The formal development is hosted at the GitHub repository: \url{https://github.com/genetsai95/DTT-QIIRT}.}%optional, e.g. related research data, source code, ... hosted on a repository like zenodo, figshare, GitHub, ...
%\supplementdetails[linktext={opt. text shown instead of the URL}, cite=DBLP:books/mk/GrayR93, subcategory={Description, Subcategory}, swhid={Software Heritage Identifier}]{General Classification (e.g. Software, Dataset, Model, ...)}{URL to related version} %linktext, cite, and subcategory are optional
\acknowledgements{We'd like to thank Fredrik Nordvall Forsberg, Hsiang-Shang Ko, and Meven Lennon-Bertrand.}%optional

\nolinenumbers %uncomment to disable line numbering

%Editor-only macros:: begin (do not touch as author)%%%%%%%%%%%%%%%%%%%%%%%%%%%%%%%%%%
\EventEditors{John Q. Open and Joan R. Access}
\EventNoEds{2}
\EventLongTitle{42nd Conference on Very Important Topics (CVIT 2016)}
\EventShortTitle{CVIT 2016}
\EventAcronym{CVIT}
\EventYear{2016}
\EventDate{December 24--27, 2016}
\EventLocation{Little Whinging, United Kingdom}
\EventLogo{}
\SeriesVolume{42}
\ArticleNo{23}
%%%%%%%%%%%%%%%%%%%%%%%%%%%%%%%%%%%%%%%%%%%%%%%%%%%%%%

\usepackage[obeyFinal,textsize=footnotesize]{todonotes}
\usepackage{manfnt}
\newcommand{\danger}{\marginpar[\hfill\dbend]{\dbend\hfill}}
\newcommand{\LT}[2][]{\todo[inline,author={L-T},caption={},#1]{#2}}

\usepackage{xspace,bbold}
\usepackage{mathtools}

\newcommand{\mathsc}[1]{\textnormal{\textsc{#1}}}

\newcommand{\Agda}{\textsc{Agda}\xspace}
\newcommand{\Coq}{\textsc{Coq}\xspace}
\newcommand{\Lean}{\textsc{Lean}\xspace}
\newcommand{\Dedukti}{\textsf{Dedukti}\xspace}
\newcommand{\AxiomK}{Axiom~\textsf{K}\xspace}

\newcommand{\xto}[1]{\xrightarrow{#1}}
\newcommand{\comp}{\circ} % function composition
  
\mathchardef\mhyphen="2D % hyphen in mathmode

\newcommand{\Ty}{\ensuremath{\mathsf{Ty}}\xspace}
\newcommand{\Tm}{\ensuremath{\mathsf{Tm}}\xspace}
\newcommand{\Ctx}{\ensuremath{\mathsf{Ctx}}\xspace}
\newcommand{\Tel}{\ensuremath{\mathsf{Tel}}\xspace}
\newcommand{\Sub}{\ensuremath{\mathsf{Sub}}\xspace}
\newcommand{\Set}{\ensuremath{\mathsf{Set}}\xspace}
\newcommand{\El}{\ensuremath{\mathsf{El}}\xspace}
\newcommand{\Id}[3]{\ensuremath{#2 \;=_{#1}\;#3}\xspace}
\newcommand{\elim}{\ensuremath{\mathsf{elim}}\xspace}
\newcommand{\implicit}[1]{{\color{gray}\{#1\}}}

\newcommand{\IR}{\ensuremath{\mathsf{QIIR}}\xspace}
\newcommand{\I}{\ensuremath{\mathsf{QII}}\xspace}
\newcommand{\emptytel}{\ensuremath{\cdot}}

\newcommand{\reduce}{\mathbin{\Rightarrow}}
\newcommand\dplus{\ensuremath{\mathbin{+\mkern-10mu+}}}

\newcommand{\transp}{\ensuremath{\mathsf{transport}}\xspace}
\newcommand{\alert}[1]{{\color{red}#1}}

% Shamelessly copied from the HoTT book

\newcommand{\nameless}{\mathord{\hspace{1pt}\underline{\hspace{1ex}}\hspace{1pt}}}

%%% Function extensionality
\newcommand{\funext}{\mathsf{funext}}
\newcommand{\happly}{\mathsf{happly}}


%%%% MACROS FOR NOTATION %%%%
% Use these for any notation where there are multiple options.

\newcommand{\blank}{\mathord{\hspace{1pt}\text{--}\hspace{1pt}}} 
%%% Definitional equality (used infix) %%%
\newcommand{\jdeq}{\equiv}      % An equality judgment
\let\judgeq\jdeq
%\newcommand{\defeq}{\coloneqq}  % An equality currently being defined
\newcommand{\defeq}{\vcentcolon\equiv}  % A judgmental equality currently being defined

%%% Term being defined
\newcommand{\define}[1]{\textbf{#1}}

%%% Vec (for example)

\newcommand{\Vect}{\ensuremath{\mathsf{Vec}}}
\newcommand{\Fin}{\ensuremath{\mathsf{Fin}}}
\newcommand{\fmax}{\ensuremath{\mathsf{fmax}}}
\newcommand{\seq}[1]{\langle #1\rangle}

%%% Dependent products %%%
\def\prdsym{\textstyle\prod}
%% Call the macro like \prd{x,y:A}{p:x=y} with any number of
%% arguments.  Make sure that whatever comes *after* the call doesn't
%% begin with an open-brace, or it will be parsed as another argument.
\makeatletter
% Currently the macro is configured to produce
%     {\textstyle\prod}(x:A) \; {\textstyle\prod}(y:B),{\ }
% in display-math mode, and
%     \prod_{(x:A)} \prod_{y:B}
% in text-math mode.
% \def\prd#1{\@ifnextchar\bgroup{\prd@parens{#1}}{%
%     \@ifnextchar\sm{\prd@parens{#1}\@eatsm}{%
%         \prd@noparens{#1}}}}
\def\prd#1{\@ifnextchar\bgroup{\prd@parens{#1}}{%
    \@ifnextchar\sm{\prd@parens{#1}\@eatsm}{%
    \@ifnextchar\prd{\prd@parens{#1}\@eatprd}{%
    \@ifnextchar\;{\prd@parens{#1}\@eatsemicolonspace}{%
    \@ifnextchar\\{\prd@parens{#1}\@eatlinebreak}{%
    \@ifnextchar\narrowbreak{\prd@parens{#1}\@eatnarrowbreak}{%
      \prd@noparens{#1}}}}}}}}
\def\prd@parens#1{\@ifnextchar\bgroup%
  {\mathchoice{\@dprd{#1}}{\@tprd{#1}}{\@tprd{#1}}{\@tprd{#1}}\prd@parens}%
  {\@ifnextchar\sm%
    {\mathchoice{\@dprd{#1}}{\@tprd{#1}}{\@tprd{#1}}{\@tprd{#1}}\@eatsm}%
    {\mathchoice{\@dprd{#1}}{\@tprd{#1}}{\@tprd{#1}}{\@tprd{#1}}}}}
\def\@eatsm\sm{\sm@parens}
\def\prd@noparens#1{\mathchoice{\@dprd@noparens{#1}}{\@tprd{#1}}{\@tprd{#1}}{\@tprd{#1}}}
% Helper macros for three styles
\def\lprd#1{\@ifnextchar\bgroup{\@lprd{#1}\lprd}{\@@lprd{#1}}}
\def\@lprd#1{\mathchoice{{\textstyle\prod}}{\prod}{\prod}{\prod}({\textstyle #1})\;}
\def\@@lprd#1{\mathchoice{{\textstyle\prod}}{\prod}{\prod}{\prod}({\textstyle #1}),\ }
\def\tprd#1{\@tprd{#1}\@ifnextchar\bgroup{\tprd}{}}
\def\@tprd#1{\mathchoice{{\textstyle\prod_{(#1)}}}{\prod_{(#1)}}{\prod_{(#1)}}{\prod_{(#1)}}}
\def\dprd#1{\@dprd{#1}\@ifnextchar\bgroup{\dprd}{}}
\def\@dprd#1{\prod_{(#1)}\,}
\def\@dprd@noparens#1{\prod_{#1}\,}

% Look through spaces and linebreaks
\def\@eatnarrowbreak\narrowbreak{%
  \@ifnextchar\prd{\narrowbreak\@eatprd}{%
    \@ifnextchar\sm{\narrowbreak\@eatsm}{%
      \narrowbreak}}}
\def\@eatlinebreak\\{%
  \@ifnextchar\prd{\\\@eatprd}{%
    \@ifnextchar\sm{\\\@eatsm}{%
      \\}}}
\def\@eatsemicolonspace\;{%
  \@ifnextchar\prd{\;\@eatprd}{%
    \@ifnextchar\sm{\;\@eatsm}{%
      \;}}}

%%% Lambda abstractions.
% Each variable being abstracted over is a separate argument.  If
% there is more than one such argument, they *must* be enclosed in
% braces.  Arguments can be untyped, as in \lam{x}{y}, or typed with a
% colon, as in \lam{x:A}{y:B}. In the latter case, the colons are
% automatically noticed and (with current implementation) the space
% around the colon is reduced.  You can even give more than one variable
% the same type, as in \lam{x,y:A}.
\def\lam#1{{\lambda}\@lamarg#1:\@endlamarg\@ifnextchar\bgroup{.\,\lam}{.\,}}
\def\@lamarg#1:#2\@endlamarg{\if\relax\detokenize{#2}\relax #1\else\@lamvar{\@lameatcolon#2},#1\@endlamvar\fi}
\def\@lamvar#1,#2\@endlamvar{(#2\,{:}\,#1)}
% \def\@lamvar#1,#2{{#2}^{#1}\@ifnextchar,{.\,{\lambda}\@lamvar{#1}}{\let\@endlamvar\relax}}
\def\@lameatcolon#1:{#1}
\let\lamt\lam
% This version silently eats any typing annotation.
\def\lamu#1{{\lambda}\@lamuarg#1:\@endlamuarg\@ifnextchar\bgroup{.\,\lamu}{.\,}}
\def\@lamuarg#1:#2\@endlamuarg{#1}

%%% Dependent products written with \forall, in the same style
\def\fall#1{\forall (#1)\@ifnextchar\bgroup{.\,\fall}{.\,}}

%%% Existential quantifier %%%
\def\exis#1{\exists (#1)\@ifnextchar\bgroup{.\,\exis}{.\,}}

%%% Dependent sums %%%
\def\smsym{\textstyle\sum}
% Use in the same way as \prd
\def\sm#1{\@ifnextchar\bgroup{\sm@parens{#1}}{%
    \@ifnextchar\prd{\sm@parens{#1}\@eatprd}{%
    \@ifnextchar\sm{\sm@parens{#1}\@eatsm}{%
    \@ifnextchar\;{\sm@parens{#1}\@eatsemicolonspace}{%
    \@ifnextchar\\{\sm@parens{#1}\@eatlinebreak}{%
    \@ifnextchar\narrowbreak{\sm@parens{#1}\@eatnarrowbreak}{%
        \sm@noparens{#1}}}}}}}}
\def\sm@parens#1{\@ifnextchar\bgroup%
  {\mathchoice{\@dsm{#1}}{\@tsm{#1}}{\@tsm{#1}}{\@tsm{#1}}\sm@parens}%
  {\@ifnextchar\prd%
    {\mathchoice{\@dsm{#1}}{\@tsm{#1}}{\@tsm{#1}}{\@tsm{#1}}\@eatprd}%
    {\mathchoice{\@dsm{#1}}{\@tsm{#1}}{\@tsm{#1}}{\@tsm{#1}}}}}
\def\@eatprd\prd{\prd@parens}
\def\sm@noparens#1{\mathchoice{\@dsm@noparens{#1}}{\@tsm{#1}}{\@tsm{#1}}{\@tsm{#1}}}
\def\lsm#1{\@ifnextchar\bgroup{\@lsm{#1}\lsm}{\@@lsm{#1}}}
\def\@lsm#1{\mathchoice{{\textstyle\sum}}{\sum}{\sum}{\sum}({\textstyle #1})\;}
\def\@@lsm#1{\mathchoice{{\textstyle\sum}}{\sum}{\sum}{\sum}({\textstyle #1}),\ }
\def\tsm#1{\@tsm{#1}\@ifnextchar\bgroup{\tsm}{}}
\def\@tsm#1{\mathchoice{{\textstyle\sum_{(#1)}}}{\sum_{(#1)}}{\sum_{(#1)}}{\sum_{(#1)}}}
\def\dsm#1{\@dsm{#1}\@ifnextchar\bgroup{\dsm}{}}
\def\@dsm#1{\sum_{(#1)}\,}
\def\@dsm@noparens#1{\sum_{#1}\,}

% Other notations related to dependent sums
%\let\setof\Set    % from package 'braket', write \setof{ x:A | P(x) }.
\newcommand{\pair}{\ensuremath{\mathsf{pair}}\xspace}
\newcommand{\tup}[2]{(#1,#2)}
\newcommand{\proj}[1]{\ensuremath{\mathsf{pr}_{#1}}\xspace}
\newcommand{\fst}{\ensuremath{\proj1}\xspace}
\newcommand{\snd}{\ensuremath{\proj2}\xspace}
\newcommand{\ac}{\ensuremath{\mathsf{ac}}\xspace} % not needed in symbol index

%%% recursor and induction
\newcommand{\rec}[1]{\mathsf{rec}_{#1}}
\newcommand{\ind}[1]{\mathsf{ind}_{#1}}
\newcommand{\indid}[1]{\ind{=_{#1}}} % (Martin-Lof) path induction principle for identity types
\newcommand{\indidb}[1]{\ind{=_{#1}}'} % (Paulin-Mohring) based path induction principle for identity types

%%% Uniqueness principles
\newcommand{\uniq}[1]{\mathsf{uniq}_{#1}}

% Paths in pairs
\newcommand{\pairpath}{\ensuremath{\mathsf{pair}^{\mathord{=}}}\xspace}
% \newcommand{\projpath}[1]{\proj{#1}^{\mathord{=}}}
\newcommand{\projpath}[1]{\ensuremath{\apfunc{\proj{#1}}}\xspace}
\newcommand{\pairct}{\ensuremath{\mathsf{pair}^{\mathord{\ct}}}\xspace}

%%% For quotients %%%
%\newcommand{\pairr}[1]{{\langle #1\rangle}}
\newcommand{\pairr}[1]{{\mathopen{}(#1)\mathclose{}}}
\newcommand{\Pairr}[1]{{\mathopen{}\left(#1\right)\mathclose{}}}

\newcommand{\im}{\ensuremath{\mathsf{im}}} % the image

%%% 2D path operations
\newcommand{\leftwhisker}{\mathbin{{\ct}_{\mathsf{l}}}}  % was \ell
\newcommand{\rightwhisker}{\mathbin{{\ct}_{\mathsf{r}}}} % was r
\newcommand{\hct}{\star}

%%% Identity types %%%
\newcommand{\idsym}{{=}}
\newcommand{\id}[3][]{\ensuremath{#2 =_{#1} #3}\xspace}
\newcommand{\idtype}[3][]{\ensuremath{\mathsf{Id}_{#1}(#2,#3)}\xspace}
\newcommand{\idtypevar}[1]{\ensuremath{\mathsf{Id}_{#1}}\xspace}
% A propositional equality currently being defined
\newcommand{\defid}{\coloneqq}

%%% Dependent paths
\newcommand{\dpath}[4]{#3 =^{#1}_{#2} #4}

%%% Reflexivity terms %%%
% \newcommand{\reflsym}{{\mathsf{refl}}}
\newcommand{\refl}[1]{\ensuremath{\mathsf{refl}_{#1}}\xspace}

%%% Path concatenation (used infix, in diagrammatic order) %%%
\newcommand{\ct}{%
  \mathchoice{\mathbin{\raisebox{0.5ex}{$\displaystyle\centerdot$}}}%
             {\mathbin{\raisebox{0.5ex}{$\centerdot$}}}%
             {\mathbin{\raisebox{0.25ex}{$\scriptstyle\,\centerdot\,$}}}%
             {\mathbin{\raisebox{0.1ex}{$\scriptscriptstyle\,\centerdot\,$}}}
}

%%% Path reversal %%%
\newcommand{\opp}[1]{\mathord{{#1}^{-1}}}
\let\rev\opp

%%% Coherence paths %%%
\newcommand{\ctassoc}{\mathsf{assoc}} % associativity law

%%% Transport (covariant) %%%
\newcommand{\trans}[2]{\ensuremath{{#1}_{*}\mathopen{}\left({#2}\right)\mathclose{}}\xspace}
\let\Trans\trans
%\newcommand{\Trans}[2]{\ensuremath{{#1}_{*}\left({#2}\right)}\xspace}
\newcommand{\transf}[1]{\ensuremath{{#1}_{*}}\xspace} % Without argument
%\newcommand{\transport}[2]{\ensuremath{\mathsf{transport}_{*} \: {#2}\xspace}}
\newcommand{\transfib}[3]{\ensuremath{\mathsf{transport}^{#1}({#2},{#3})\xspace}}
\newcommand{\Transfib}[3]{\ensuremath{\mathsf{transport}^{#1}\Big(#2,\, #3\Big)\xspace}}
\newcommand{\transfibf}[1]{\ensuremath{\mathsf{transport}^{#1}\xspace}}

%%% 2D transport
\newcommand{\transtwo}[2]{\ensuremath{\mathsf{transport}^2\mathopen{}\left({#1},{#2}\right)\mathclose{}}\xspace}

%%% Constant transport
\newcommand{\transconst}[3]{\ensuremath{\mathsf{transportconst}}^{#1}_{#2}(#3)\xspace}
\newcommand{\transconstf}{\ensuremath{\mathsf{transportconst}}\xspace}

%%% Map on paths %%%
\newcommand{\mapfunc}[1]{\ensuremath{\mathsf{ap}_{#1}}\xspace} % Without argument
\newcommand{\map}[2]{\ensuremath{{#1}\mathopen{}\left({#2}\right)\mathclose{}}\xspace}
\let\Ap\map
%\newcommand{\Ap}[2]{\ensuremath{{#1}\left({#2}\right)}\xspace}
\newcommand{\mapdepfunc}[1]{\ensuremath{\mathsf{apd}_{#1}}\xspace} % Without argument
% \newcommand{\mapdep}[2]{\ensuremath{{#1}\llparenthesis{#2}\rrparenthesis}\xspace}
\newcommand{\mapdep}[2]{\ensuremath{\mapdepfunc{#1}\mathopen{}\left(#2\right)\mathclose{}}\xspace}
\let\apfunc\mapfunc
\let\ap\map
\let\apdfunc\mapdepfunc
\let\apd\mapdep

%%% 2D map on paths
\newcommand{\aptwofunc}[1]{\ensuremath{\mathsf{ap}^2_{#1}}\xspace}
\newcommand{\aptwo}[2]{\ensuremath{\aptwofunc{#1}\mathopen{}\left({#2}\right)\mathclose{}}\xspace}
\newcommand{\apdtwofunc}[1]{\ensuremath{\mathsf{apd}^2_{#1}}\xspace}
\newcommand{\apdtwo}[2]{\ensuremath{\apdtwofunc{#1}\mathopen{}\left(#2\right)\mathclose{}}\xspace}

%%% Identity functions %%%
\newcommand{\idfunc}[1][]{\ensuremath{\mathsf{id}_{#1}}\xspace}

%%% Other meanings of \sim
\newcommand{\bisim}{\sim}       % bisimulation
\newcommand{\eqr}{\sim}         % an equivalence relation

%%% Equivalence types %%%
\newcommand{\eqv}[2]{\ensuremath{#1 \simeq #2}\xspace}
\newcommand{\eqvspaced}[2]{\ensuremath{#1 \;\simeq\; #2}\xspace}
\newcommand{\eqvsym}{\simeq}    % infix symbol
\newcommand{\texteqv}[2]{\ensuremath{\mathsf{Equiv}(#1,#2)}\xspace}
\newcommand{\isequiv}{\ensuremath{\mathsf{isequiv}}}
\newcommand{\qinv}{\ensuremath{\mathsf{qinv}}}
\newcommand{\ishae}{\ensuremath{\mathsf{ishae}}}
\newcommand{\linv}{\ensuremath{\mathsf{linv}}}
\newcommand{\rinv}{\ensuremath{\mathsf{rinv}}}
\newcommand{\biinv}{\ensuremath{\mathsf{biinv}}}
\newcommand{\lcoh}[3]{\mathsf{lcoh}_{#1}(#2,#3)}
\newcommand{\rcoh}[3]{\mathsf{rcoh}_{#1}(#2,#3)}
\newcommand{\hfib}[2]{{\mathsf{fib}}_{#1}(#2)}

%%% Map on total spaces %%%
\newcommand{\total}[1]{\ensuremath{\mathsf{total}(#1)}}

%%% Universe types %%%
%\newcommand{\type}{\ensuremath{\mathsf{Type}}\xspace}
\newcommand{\UU}{\ensuremath{\mathcal{U}}\xspace}
% Universes of truncated types
\newcommand{\typele}[1]{\ensuremath{{#1}\text-\mathsf{Type}}\xspace}
\newcommand{\typeleU}[1]{\ensuremath{{#1}\text-\mathsf{Type}_\UU}\xspace}
\newcommand{\typelep}[1]{\ensuremath{{(#1)}\text-\mathsf{Type}}\xspace}
\newcommand{\typelepU}[1]{\ensuremath{{(#1)}\text-\mathsf{Type}_\UU}\xspace}
\let\ntype\typele
\let\ntypeU\typeleU
\let\ntypep\typelep
\let\ntypepU\typelepU
%\renewcommand{\set}{\ensuremath{\mathsf{Set}}\xspace}
%\newcommand{\setU}{\ensuremath{\mathsf{Set}_\UU}\xspace}
\newcommand{\prop}{\ensuremath{\mathsf{Prop}}\xspace}
\newcommand{\propU}{\ensuremath{\mathsf{Prop}_\UU}\xspace}
%Pointed types
\newcommand{\pointed}[1]{\ensuremath{#1_\bullet}}

%
%%% The empty type
\newcommand{\emptyt}{\ensuremath{\mathbf{0}}\xspace}

%%% The unit type
\newcommand{\unit}{\ensuremath{\mathbf{1}}\xspace}
\newcommand{\ttt}{\ensuremath{\star}\xspace}

%%% The two-element type
\newcommand{\bool}{\ensuremath{\mathbf{2}}\xspace}
\newcommand{\btrue}{{1_{\bool}}}
\newcommand{\bfalse}{{0_{\bool}}}
\newcommand{\belim}{{\elim_{\bool}}}

%%% Injections into binary sums and pushouts
\newcommand{\inlsym}{{\mathsf{inl}}}
\newcommand{\inrsym}{{\mathsf{inr}}}
\newcommand{\inl}{\ensuremath\inlsym\xspace}
\newcommand{\inr}{\ensuremath\inrsym\xspace}

%%% Natural numbers
\newcommand{\N}{\ensuremath{\mathbb{N}}\xspace}
%\newcommand{\N}{\textbf{N}}
\let\nat\N
\newcommand{\natp}{\ensuremath{\nat'}\xspace} % alternative nat in induction chapter

\newcommand{\zerop}{\ensuremath{0'}\xspace}   % alternative zero in induction chapter
\newcommand{\suc}{\mathsf{suc}}
\newcommand{\sucp}{\ensuremath{\suc'}\xspace} % alternative suc in induction chapter
\newcommand{\add}{\mathsf{add}}
\newcommand{\ack}{\mathsf{ack}}
\newcommand{\ite}{\mathsf{iter}}
\newcommand{\assoc}{\mathsf{assoc}}
\newcommand{\dbl}{\ensuremath{\mathsf{double}}}
\newcommand{\dblp}{\ensuremath{\dbl'}\xspace} % alternative double in induction chapter


% Join lists

\newcommand{\JList}[1]{\mathsf{JList}\,#1}
\newcommand{\List}[1]{\mathsf{List}\,#1}
\newcommand{\flatten}{\mathsf{flatten}}
\newcommand{\Jnil}{\mathsf{[]}}
\newcommand{\Jsing}[1]{\mathsf{[#1]}}
\newcommand{\Jconcat}[2]{#1 \mathop{+\!\!+} #2}

% Function definition with domain and codomain
\newcommand{\function}[4]{\left\{\begin{array}{rcl}#1 &
      \longrightarrow & #2 \\ #3 & \longmapsto & #4 \end{array}\right.}


%%% Sets
\newcommand{\bin}{\ensuremath{\mathrel{\widetilde{\in}}}}

%%% Macros for the formal type theory
\newcommand{\emptyctx}{\ensuremath{\cdot}}
\newcommand{\emptysub}{\ensuremath{\cdot}}
\newcommand{\ctx}{\ensuremath{\mathsf{ctx}}}
\newcommand{\idS}{\ensuremath{\mathsf{id}}} 
%\newcommand{\instSub}[1]{\ensuremath{(\idS, #1)}}
\newcommand{\instSub}[1]{\ensuremath{\langle\,#1\,\rangle}}

%\newcommand{\sub}[2]{[#1]\,{#2}}
\newcommand{\sub}[2]{{{#1}\,[#2]}}
\newcommand{\subTm}[2]{\sub{#1}{#2}_{\Tm}}
\newcommand{\subTy}[2]{\sub{#1}{#2}_{\Ty}}
\newcommand{\subTel}[2]{\sub{#1}{#2}_{\Tel}}
\newcommand{\subM}[2]{{#1}\,[#2]^M}
\newcommand{\subTmM}[2]{\subM{#1}{#2}_{\Tm}}
\newcommand{\subTyM}[2]{\subM{#1}{#2}_{\Ty}}
\newcommand{\subTelM}[2]{\subM{#1}{#2}_{\Tel}}

\newcommand{\cc}{\mathsf{c}}



\begin{document}

\maketitle

\begin{abstract}
  Type theory can be defined in a type theory as a quotient inductive-inductive type, but well-typed terms are littered with explicit coercions, i.e.\ \transp's, along equality constructors and proofs need to take account of coercions painstakingly alongside the interesting part.
  This mess has been dubbed the `transport hell' but typically suppressed in presentation for clarity, hiding the gap between the intention and the formalisation of type theory in type theory.
  In this paper, we aim for shortening the gap using quotient inductive-inductive-recursive types and definitions by overlapping patterns altogether, reducing the use of \transp's.
  As a case study, we investigate (parallel) substitution calculus and type theory with a universe and $\Pi$-types as quotient inductive-inductive-recursive types and compare ours with quotient inductive-inductive definitions.
\end{abstract}

\section{Introduction} \label{sec:intro}
\LT{%
Points to elaborate:
\begin{enumerate}
  \item QIITs are littered with transports, causing too much pains while formalising a type theory.
  \item Rewrite rules are popular to make structural rules definitional but requires efforts to justify meta-theoretic properties in general.
  \item Further, the confluence check adopted by \Agda is based on complete development and requires one-step parallel reductions to have a confluent term, making a definition more complicated than necessary.
  \item Moreover, the use of rewrite rules is typically \emph{not} presented and discussed in literature. We are ignoring the elephant in the room.
  \item Instead, definitions by overlapping patterns (in theory) requires strong normalisation and local confluence, making definitions simpler to design.
  \item The rules for type substitutions are \emph{structural} shared with other type theories based on parallel substitution (in particular cwf).
  
\end{enumerate}
}


\paragraph*{Contributions}
\begin{itemize}
  \item Exploration of the use of quotient inductive-inductive-recursive types and definitions by overlapping patterns.
    In particular, we give in a type theory:
    \begin{itemize}
      \item a definition of parallel substitution calculus;
      \item a definition of type theory with a universe and $\Pi$-types,
    \end{itemize}
   using quotient inductive-inductive-recursive types and definitions by overlapping patterns to define type substitution and term substitution partially.
  \item Comparison with other definitions using quotient inductive-inductive types.
\end{itemize}


\subsection{Plan of the paper}

\section{Metatheory and formalisation}

\cite{UFP2013}
\LT{
Points to include:
\begin{enumerate}
  \item identity type $x =^{A} y$ for $x, y : A$, dependent identity type, $t =^{P}_{p} u \defeq \transfib{P}{p}{t} =^{P y} u$ for $t : P(x)$ and $u : P(y)$, heterogeneous equality~\cite{McBride1999} in \Agda $x \simeq y$ for $x : A$ and $y : B$
  \item We work with intensional type theory with uniqueness of identity proof and function extensionality (only used for NbE and the standard model for the extensional identity type).
  \item We use Agda to formally implement our definitions with the following options: \texttt{-{}-with-K}, \texttt{-{}-local-confluence-check}, \texttt{-{}-exact-split}, and \texttt{-{}-rewriting} using postulated equations~\cite{Licata2011} to introduce equality constructors for quotient types.
\end{enumerate}}

\LT{The symbol \textdbend indicates a property that is proved informally or checked partially with \Agda (because of postulated quotient inductive types).}
\subsection{Inductive-recursive types}
\cite{Dybjer2003,Dybjer2000,Dybjer1999}

\LT{Mention that the recursion part of induction-recursion are already not defined by elimination rule.}

\subsection{Definitions by overlapping patterns} \label{sec:meta:overlapping}
\cite{Cockx2014,Altenkirch2016a}
\LT{We use $f\;\vec{x} \reduce \vec{t}$ to indicate a function clause of a definition by overlapping patters.}
\subsection{Quotient inductive(-inductive)-recursive types}
\LT{Give an example of QIRT: join list?}
\subsection{Definitions by rewrite rules}
\cite{Cockx2020,Cockx2021}

Unfortunately, neither general schemata of quotient inductive-inductive-recursive types nor definitions by overlapping patterns have been developed or implemented in existing proof assistants.

\paragraph*{Local confluence}
\LT{Explain the notion of critical pair; local peak}
\paragraph*{Strong normalisation}
\cite{Abel2002}
\LT{Explain how these two conditions imply global confluence}

\section{Type theories as quotient inductive-inductive-recursive types} \label{sec:QIIRTs}
\LT{
\begin{enumerate}
  \item Goal: no transports in the definition.
  \item Motivation: make type substitution definitional so that we do not have to apply transport along structural rules for types explicitly.
  \item Develop the quotient inductive-inductive-recursive definition of type theory step by step (which needs to be locally confluent and terminating).
    \begin{enumerate}
      \item parallel substitution: type substitution
      \item the type of elements of $A$: term substitution needs to be split into two definitional substitution and explicit term substitution to maintain local confluence.
      \item $\Pi$-type: because of the substitution lifting $\sigma^+ \defeq (\pi_1\idS, \pi_2\idS)$ is used we need to introduce a definitional lifting to maintain local confluence.
        We prefer categorical combinator and avoid using $\left< t \right>$ in the definition, which would introduce transports in the definition.
      \item Other type formers can be introduced as usual (as long as we use categorical combinators).
    \end{enumerate}
\end{enumerate}
}

We begin with the definition of (parallel) substitution calculus~\cite{Martin-Lof1992} as a quotient inductive-inductive type, emphasising how the use of transports in its very definition complicates formal reasoning.
Then we introduce its quotient inductive-inductive-recursive counterpart (\cref{subsec:SC-QIIRT}) which shortens the gap between formal and informal reasoning and we develop a type theory with a universe (\cref{subsec:SC+U}), $\Pi$-types (\cref{subsec:SC+U+Pi}), and other type formers (\cref{subsec:SC+U+Pi+more}).

\subsection{Substitution calculus as a quotient inductive-inductive type} \label{subsec:SC-QIIT}

Substitution calculus has a type $\Ctx$ of contexts, a type $\Ty\,\Gamma$ of types under some context $\Gamma : \Ctx$, a type $\Sub\;\Gamma\;\Delta$ of substitutions from the domain $\Gamma$ and the codomain $\Delta$, and a type $\Tm\;\Gamma\;A$ of terms under a context $\Gamma$ and its type $A$. 
These types amount to the following types indexed by types being defined (hence inductive-inductive):
\begin{alignat*}{3}
  \Ctx   & : \Set                   \\
  \Ty    & : \Ctx \to \Set          \\
  \Sub   & : \Ctx \to \Ctx \to \Set \\
  \Tm    & : (\Gamma : \Ctx) \to \Ty\;\Gamma \to \Set
\end{alignat*}
The type $\Ctx$ has two constructors $\emptyctx$ for the empty context and $\blank,\blank$ for context extension:
\begin{alignat*}{3}
  \emptyctx & : \Ctx \\
  \blank,\blank & : (\Gamma : \Ctx) \to \Ty\;\Gamma \to \Ctx
\end{alignat*}
where the context extension $\blank,\blank$ requires that the type $\Ty$ indexed by a context $\Gamma$, underscoring the nature of inductive-inductive definition.

Type substitution takes a substitution $\sigma : \Sub\;\Gamma\;\Delta$ and an inhabitant $A : \Ty\;\Delta$ to form a type under $\Gamma$, so type substitution as a constructor has the type
\begin{alignat*}{3}
  [\blank]\blank &: \implicit{\Gamma, \Delta} \; \Sub\;\Gamma\;\Delta \to \Ty\;\Delta \to \Ty\;\Gamma.
\end{alignat*}
where $\Gamma$ and $\Delta$ (coloured {\color{gray}grey}) above are quantified implicitly.

If we were defining the notion of cwfs, type substitution would be enough for $\Ty$.
For our inductive definition (i.e.\ the initial cwf), however, we need a base case in $\Ty\;\Gamma$, otherwise the type $\Ty$ would be empty as well as other types.
Hence another constructor is introduced:
\begin{alignat*}{3}
  \UU & : \implicit{\Gamma} & \Ty\, \Gamma.
\end{alignat*}
For now, $\UU$ serves as a constant, but we will reuse $\UU$ for the type of small types later.

Substitutions from $\Gamma$ to $\Delta$ can be understood intuitively as lists of terms of type $A$ under the context $\Gamma$ for each $A$ in $\Delta$, so we have the empty substitution $\emptysub$ and substitution extensions $\sigma, t$ by some term~$t$:
\begin{alignat*}{3}
  \emptysub & : \implicit{\Gamma}\;\Sub\;\Gamma\;\emptyctx \\
  \blank,\blank & : \implicit{\Gamma, \Delta, A}\;(\sigma : \Sub\,\Gamma\,\Delta) \to \Tm\;\Gamma\;([ \sigma ]\, A) \to \Sub\;\Gamma\;(\Delta, A),
\end{alignat*}
Note that type substitution $[\sigma]\;A$ is needed, because $A$ is well-formed under the context $\Delta$ instead of $\Gamma$.
As substitution calculus is the initial cwf with a constant type $\UU$, substitutions also have the identity substitution $\idS$ and composition $\blank;\blank$
\begin{alignat*}{3}
  \idS & : \implicit{\Gamma}\;\Sub\;\Gamma\;\Gamma \\
  \blank;\blank & : \implicit{\Gamma, \Delta, \Theta}\;\Sub\;\Gamma\;\Delta \to \Sub\;\Delta\;\Theta \to \Sub\;\Gamma\;\Theta
\end{alignat*}
satisfying certain laws (introduced later).
Context comprehension is given by projections from \emph{non-empty} substitutions $\Sub\;\Gamma\;(\Delta, A)$:
\begin{alignat*}{3}
  \pi_1 & : \implicit{\Gamma, \Delta, A}\;\Sub\;\Gamma\;(\Delta, A) \to \Sub\;\Gamma\;\Delta \\
  \pi_2 & : \implicit{\Gamma, \Delta, A}\;(\sigma : \Sub\;\Gamma\;(\Delta, A)) \to \Tm\;\Gamma\;([ \pi_1\,\sigma ]\; A)
\end{alignat*}
where $\pi_1\,\sigma$ and $\pi_2\,\sigma$ can be intuitively understood as the \emph{tail} and the \emph{head} of a non-empty substitution $\sigma : \Sub\;\Gamma\;(\Delta, A)$ respectively.
Again, type substitution $[\pi_1\,\sigma]\,A$ is needed for~$\pi_2$, because $A$ is well-formed under the context $\Delta$ instead of $\Gamma$.
Finally, we also have term substitution $[\sigma]\,t$ whose type is $[\sigma]\,A$ for a term $t : \Tm\;\Delta\;A$
\begin{alignat*}{3}
  [\blank] \blank & : \implicit{\Gamma,\Delta, A}\;(\sigma : \Sub\;\Gamma\;\Delta) \to \Tm\;\Delta\;A \to \Tm\;\Gamma\;([\sigma]\; A)
\end{alignat*}
where the symbol $[\blank]\blank$ is overloaded.

The structural rules for type substitution are stipulated by following equality constructors:
\begin{alignat*}{3}
  [\idS]_T & : \implicit{\Gamma, A}                               && [ \idS ] \;A         && =^{\Ty\,\Gamma}\;A \\
  [;]_T    & : \implicit{\Gamma, \Delta, \Theta, \sigma, \tau, A} && [ \sigma ; \tau ]\;A && =^{\Ty\,\Gamma} [ \sigma ]\;([ \tau ]\;A) \\
  []\UU      & : \implicit{\Gamma, \Delta, \sigma}                && [ \sigma ]\;\UU        && =^{\Ty\,\Gamma} \UU
\end{alignat*}

The codomain $\Delta$ of a substitution $\sigma:\Sub\;\Gamma\;\Delta$ also tells the length of terms in $\sigma$, giving rise two $\eta$-laws in the case that $\Delta$ is empty and non-empty:
\begin{alignat*}{5}
  \emptyctx\eta   & : \implicit{\Gamma, A, \sigma} && \sigma        && =^{\Sub\,\Gamma\,\emptyctx} & \emptysub \\
  \pi\eta         & : \implicit{\Gamma, \Delta, \sigma} && \sigma   && =^{\Sub\,\Gamma\,(\Delta, A)} &&  (\pi_1 \sigma, \pi_2 \sigma)
\end{alignat*}

The laws for substitution composition are stipulated using equality constructors.
\begin{alignat*}{5}
  \mathsf{idr}    & : \implicit{\Gamma, \Delta, \sigma} && {\sigma ; \idS_{\Delta}} && =^{\Sub\,\Gamma\,\Delta} && {\sigma} \\
  \mathsf{idl}    & : \implicit{\Gamma, \Delta, \sigma} && {\idS_{\Gamma} ; \sigma} && =^{\Sub\,\Gamma\,\Delta} && {\sigma} \\
  ;\text{-}\mathsf{assoc} & : \implicit{\Gamma, \Delta, \Theta, \Xi, \sigma, \tau, \gamma} && (\sigma ; \tau) ; \gamma && =^{\Sub\,\Gamma\,\Theta} &&  \sigma ; (\tau ; \gamma) \\
  \mathsf{concat} & : \implicit{\Gamma, \Delta, \Theta, \sigma, \tau, A, t} && \sigma ; (\tau , t)      && =^{\Sub\,\Gamma\,(\Theta, A)} &&  (\sigma ; \tau) , \alert{\transfib{\Tm\,\Gamma}{[;]_{\Ty}^{-1}}{\color{black}[ \sigma ] t}}
\end{alignat*}
Substitution composition $\blank;\blank$ acts like a list concatenation, so $\sigma; (\tau, t)$ is intuitively equal to $(\sigma; \tau), [\sigma]\,t$ whereas the term $t$ is of type $\Tm\;\Delta\;([\tau]\;A)$. 
However, by the type of substitution extension, $[\sigma]\,t$ requires to be of type $\Tm\;\Gamma\;([\sigma; \tau]\,A)$ instead, so we have to transport $[\sigma]\,t$ explicitly along $[;]_{\Ty}^{-1}\colon [\sigma]\,[\tau]\;A = [\sigma;\tau]\,A$ to obtain a term of type $\Tm\;\Gamma\;([\sigma;\tau]\;A)$.

The tail and the head of a non-empty substitution $(\sigma, t)$ is obviously $\sigma$ and $t$ respectively:
\begin{alignat*}{5}
  \pi_1\beta      & : \implicit{\Gamma, \Delta, \Theta, \sigma, A, t} && \pi_1(\sigma , t)        && =^{\Sub\,\Gamma,\Delta} &&  \sigma \\
  \pi_2\beta      & : \implicit{\Gamma, \Delta, \Theta, \sigma, A, t} && \pi_2(\sigma , t)        && =^{\Tm\,\Gamma\,([\blank]\,A)}_{\alert{\pi_1\beta}} &&  t, 
\end{alignat*}
whereas $\pi_2(\sigma, t)$ is an habitant of $\Tm\;\Gamma\;([\pi_1\,(\sigma, t)]\,A)$ instead of $\Tm\;\Gamma\;([\sigma]\,A)$ on the right-hand side. 
Therefore, we have to transport $\pi_2(\sigma, t)$ along $\pi_1\beta$, so the above equality constructor is, in fact, $\transfib{\Tm\;\Gamma\;([\blank]\,A)}{\pi_1\beta}{\pi_2(\sigma, t)} =^{\Tm\;\Gamma\;([\sigma]\,A)} t$.

Similarly, for term substitution, terms are transported along the corresponding rules:
\begin{alignat*}{5}
  [\idS]_t         & : \implicit{\Gamma, A, t} && {[\idS]\,t}         && =^{\Tm\;\Gamma}_{\alert{[\idS]_\Ty}}  && t \\
  [;]_t            & : \implicit{\Gamma, \Delta, \Theta, \sigma, \tau, t} && {[\sigma ; \tau]\,t} && =^{\Tm\;\Gamma}_{\alert{[;]_{\Ty}}}   && {[ \sigma ]\,[ \tau ]\,t}
\end{alignat*}
equivalent to the following (homogeneous) identities
\[
  \transfib{\Tm\;\Gamma}{[\idS]_{\Ty}}{[\idS]\,t} =^{\Tm\,\Gamma\,A} t
  \quad\text{and}\quad
  \transfib{\Tm\;\Gamma}{[;]_{\Ty}}{[\sigma;\tau]\,t} =^{\Tm\,\Gamma\,([\sigma]\,[\tau]A)} [\sigma]\,[\tau]\,t
\]
respectively.
Constructors introduced so far complete the definition of substitution calculus.

The use of transports in the formal definition fixes type mismatches but hinders equational reasoning about these terms even for a simple fact below.
\begin{example}\label{ex:pi2-comp}
  Given substitutions $\sigma : \Sub\;\Gamma\; \Delta$ and $\tau : \Sub\;\Delta\;(\Theta, A)$ for any $A : \Ty\,\Theta$, we may apply the projection $\pi_2$ to the composite $(\sigma; \tau)$ to access the first term $\pi_2(\sigma; \tau)$ of type $[\sigma;\tau] A$ under the context $\Gamma$, and this term should be equal to the first term $\pi_2\,\tau$ of $\tau$ after applying the substitution $\sigma$. 
  In short, the following equality apparently holds
  \[
    \pi_2\,(\sigma ; \tau) = [\sigma] (\pi_2\,\tau)
  \]
  by a back-of-the-envelope calculation
  \begin{equation} \label{eq:pi2-comp-proof}
    \pi_2\,(\sigma ; \tau) 
    = \pi_2\,(\sigma; (\pi_1\,\tau, \pi_2\,\tau))
    = \pi_2\,(\sigma;\pi_1\,\tau, [\sigma]\,(\pi_2\,\tau))
    = [\sigma] (\pi_2\,\tau).
  \end{equation}
  Yet, the left-hand side is a term of type $[\pi_1\,(\sigma;\tau)] A$, but the other is $[\sigma] [\pi_1\,\tau] A$.
  Hence the above identity does not even make sense, since their types do not match.
  Alas, instead, we have to write $\pi_2\,(\sigma ; \tau) =^{\Tm\,\Gamma}_{p} [\sigma] (\pi_2\,\tau)$ or, equivalently
  \begin{equation}\label{eq:pi2-comp-real-proof}
    ([\pi_1(\sigma; \tau)]\,A, \pi_2(\sigma; \tau)) =^{(A : \Ty\,\Gamma) \times (\Tm \Gamma A)} ([\sigma]\,[\pi_1\,\tau]A, [\sigma] (\pi_2\,\tau))
  \end{equation}
  as inhabitants of a $\Sigma$-type, so we can reason about term equalities along with type equalities.
  Moreover, in~\eqref{eq:pi2-comp-proof} we have used the rule $\mathsf{concat}$ which introduced another transported term, so we will have to eliminate that $\transp$ to derive the right hand side.

  To better illustrate the annoyance, note that a complete proof of \eqref{eq:pi2-comp-real-proof} requires us to show each of following equations:
  \begin{alignat*}{3}
         & ([\pi_1(\sigma; \tau)]\,A                    &&, \pi_2\,(\sigma ; \tau)) \\
    = {} & ([\pi_1(\sigma; (\pi_1\tau , \pi_2\tau))]\,A &&, \pi_2\,(\sigma; (\pi_1\,\tau, \pi_2\,\tau))) \\
    = {} & ([\pi_1(\sigma;\pi_1\,\tau, \transfib{}{[;]_{\Ty}^{-1}}{\color{black}[ \sigma ] (\pi_2\;\tau)})]\,A &&, \pi_2\,(\sigma;\pi_1\,\tau, \transfib{}{[;]_{\Ty}^{-1}}{\color{black}[ \sigma ] (\pi_2\;\tau)})) \\
    = {} & ([\sigma;\pi_1\,\tau]\,A &&, \transfib{}{[;]_{\Ty}^{-1}}{\color{black}[ \sigma ] (\pi_2\;\tau)})) \nonumber \\
    = {} & (\sub{\sigma}{\sub{\pi_1\tau}{A}} &&, [\sigma] (\pi_2\,\tau)). \nonumber
  \end{alignat*}
  The first three equations follow from Lemma~2.3.4 in \cite{UFP2013}, while the last equation has nothing to do with substitution calculus but a property of transport (\cite[Lemma~2.3.2]{UFP2013}).
\end{example}

\subsection{Substitution calculus as a quotient inductive-inductive-recursive type} \label{subsec:SC-QIIRT}

To retain the intuitive way of reasoning such as \eqref{eq:pi2-comp-proof} \emph{formally}, we would like to make type substitution rules definitional, since the root cause is that type substitution as a constructor cannot compute so that type mismatches occur after substitution.

One possibility is to define substitution calculus as a QII\emph{R} type where type substitution, as a simultaneously defined recursion, does compute.
Now that $[ \idS ]_{\Ty}\;A = A$ and $[ \sigma ; \tau ]_{\Ty}\;A = [ \sigma ]_{\Ty}\;([ \tau ]_{\Ty}\;A)$ do not make $[\blank]_{\Ty}\blank$ a (total) function, we may define it by
\begin{alignat*}{3}
[\blank]_{\Ty} \blank &: \implicit{\Gamma, \Delta} \; \Sub\;\Gamma\;\Delta \to \Ty\;\Delta \to \Ty\;\Gamma \\
[ \sigma ]_{\Ty}\;\UU & = \UU
\end{alignat*}
Note that this definition is \emph{not} inductive-recursive in the sense of Dybjer and Setzer~\cite{Dybjer2000,Dybjer2003}: in their schema, the codomain of a simultaneously defined function does not refer to any inductive type being defined.\footnote{%
  While the general schema for this more liberal notion of induction-recursion remains underdeveloped, this definition is accepted in \Agda and used in other formalisation of type theory~\cite{Danielsson2006}.}
Still, although $[\idS]_{\Ty}\,A$ and $[\sigma; \tau]_{\Ty}\,A$ can be proved equal to~$A$ and~$[\sigma]_{\Ty}\,[\tau]_{\Ty}\,A$ respectively, they are not definitional but propositional.
This definition fails to eliminate any transport in the definition of substitution calculus.

Our second attempt is to define type substitution with additional rules:
\begin{alignat}{3}
[\blank]_{\Ty} \blank &: \implicit{\Gamma, \Delta} \; \Sub\;\Gamma\;\Delta \to \Ty\;\Delta \to \Ty\;\Gamma \nonumber \\
[ \sigma ]_{\Ty}       \;\UU  & = \UU                                   \label{eq:type-sub-at2-1} \\
[ \idS ]_{\Ty}         \;A    & = A                                     \label{eq:type-sub-at2-2} \\
[ \sigma ; \tau ]_{\Ty}\;A    & = [ \sigma ]_{\Ty}\;([ \tau ]_{\Ty}\;A) \label{eq:type-sub-at2-3}
\end{alignat}
However, this definition does \emph{not} reduce \eqref{eq:type-sub-at2-2} or \eqref{eq:type-sub-at2-3} definitionally.
By design, definitions by pattern matching will be translated to a form that corresponds to the \emph{eliminator} either using \emph{first-match semantics}~\cite{Cockx2020a} or an explicit construct for pattern matching, so only the eliminator for $\Ty\;\Delta$ will be considered in the above definition.
\LT[noinline]{How about \Coq and \Lean?}
Using the first-match semantics, the above definition amounts to the following definition:
\begin{alignat*}{3}
[\blank]_{\Ty} \blank &: \implicit{\Gamma, \Delta} \; \Sub\;\Gamma\;\Delta \to \Ty\;\Delta \to \Ty\;\Gamma \\
[ \sigma ]_{\Ty}\;\UU  & \reduce \UU
\end{alignat*}
where $\reduce$ is used to emphasise the reduction (which entails the definitional equality): we have done nothing different from our first attempt! 
%As a result, neither  are definitional.
We may swap \eqref{eq:type-sub-at2-1} with \eqref{eq:type-sub-at2-2} and \eqref{eq:type-sub-at2-3} to make \eqref{eq:type-sub-at2-2} and \eqref{eq:type-sub-at2-3} definitional, but then \eqref{eq:type-sub-at2-1} will be translated to 
\begin{alignat*}{3}
[ \emptysub ]_{\Ty}\;\UU         & \reduce \UU \\
[ \sigma, t ]_{\Ty}\;\UU         & \reduce \UU \\
[ \pi_1\,\sigma ]_{\Ty}\;\UU     & \reduce \UU,
\end{alignat*}
making $[\sigma]\;\UU = \UU$ propositional instead.
Not being able to reduce $[ \sigma ]_{\Ty}\;\UU$ to $\UU$ would make our later extension with a type $\El\,a$ for $a : \Tm\;\Gamma\;\UU$ require another transport in the definition.
Likewise, for any extension with a type former such as $\Pi$-types, structural rules would still be propositional.
That is, using the standard semantics of function definition, we are not able to make type substitution definitional on \emph{both} $\Sub\;\Gamma\;\Delta$ and $\Ty\;\Delta$.

To ensure that type substitution is definitional for all \emph{rules}, we turn to a definition by overlapping patterns (\Cref{sec:meta:overlapping}) instead to make each clause definitional. 
In addition, we can include other derived rules such as $[\pi_1(\sigma, t)]_{\Ty}\;A = [ \sigma]_{\Ty}\;A$ as long as it is locally confluent and terminating.
We end up with the following definition for type substitution:
\begin{alignat}{3}
[\blank]_{\Ty} \blank            &: \implicit{\Gamma, \Delta} \; \Sub\,\Gamma\,\Delta \to \Ty\,\Delta \to \Ty\,\Gamma \nonumber \\
[ \idS ]_{\Ty}\;A                & \reduce A \label{eq:def-type-subst-1}\\
[ \sigma ; \tau ]_{\Ty}\;A       & \reduce [ \sigma ]_{\Ty}\;([ \tau ]_{\Ty}\;A) \label{eq:def-type-subst-2}\\
[ \pi_1(\sigma, t) ]_{\Ty}\;A    & \reduce [\sigma]_\Ty\;A                       \label{eq:def-type-subst-3} \\
[ \pi_1(\sigma; \tau) ]_{\Ty}\;A & \reduce [\sigma]_\Ty\;([\pi_1\tau]_\Ty\;A)    \label{eq:def-type-subst-4} \\
[ \sigma ]_{\Ty}\;\UU            & \reduce \UU                                   \label{eq:def-type-subst-5}
\end{alignat}
We first check that this definition of type substitution is confluent.\footnote{%
\Agda implements a local confluence checker with the experimental option \text{-{}-local-confluence-check} which is turned on globally in our formalisation, but we do not trust experimental features blindly.}
\begin{proposition}[Local confluence] \label{prop:local-confluence-1}
  Type substitution $[\blank]_{\Ty}\blank$ is locally confluent.
  \danger
\end{proposition}
\begin{proof}
  To show the local confluence, we consider all local peaks, i.e.\ each combination of \eqref{eq:def-type-subst-5} with \eqref{eq:def-type-subst-1}--\eqref{eq:def-type-subst-4},
  \[
    [\idS]_{\Ty}\;\UU, \qquad [\sigma;\tau]_{\Ty}\;\UU, \qquad [\pi_1(\sigma, t)]_{\Ty}\;\UU, \qquad\text{and}\qquad [\pi_1(\sigma; \tau)]_{\Ty}\;\UU
  \]
  and show that each applicable clause does result in $\UU$.

  For example, $[\sigma;\tau]_{\Ty}\;\UU \reduce [\sigma]_{\Ty}\;([\tau]_{\Ty}\;\UU) \reduce [\sigma]_{\Ty}\;\UU \reduce \UU$ using \eqref{eq:def-type-subst-2} as the first reduction and $[\sigma;\tau]_{\Ty}\;\UU \reduce \UU$ using \eqref{eq:def-type-subst-5} as the first reduction.
  It is easy to see that remaining local peaks reduce to the same term.
\end{proof}
\begin{proposition}[Termination]
  Type substitution $[\blank]_{\Ty}\blank$ is terminating.
  \danger
\end{proposition}
\begin{proof}
  \LT{clearly terms on the RHS are structurally smaller~\cite{Abel2002}.}
  
\end{proof}


With this definition of type substitution, other equality constructors can be introduced without any $\transp$, we list updated constructors as follows.
\begin{alignat*}{5}
  \mathsf{concat} & : \implicit{\Gamma, \Delta, \Theta, \sigma, \tau, A, t} &&\sigma ; (\tau , t) && =^{\Sub\,\Gamma\,(\Theta, A)} &&  (\sigma ; \tau), [ \sigma ]\;t \\
  \pi_2\beta      & : \implicit{\Gamma, \Delta, \Theta, \sigma, A, t} && \pi_2(\sigma , t)        && =^{\Tm\,\Gamma\, A} &&  t \\
  [\idS]t         & : \implicit{\Gamma, A, t} && {[\idS]\,t}          && =^{\Tm\,\Gamma\,A} && t \\
  [;]t            & : \implicit{\Gamma, \Delta, \Theta, \sigma, \tau, t} && {[\sigma ; \tau]\,t} && =^{\Tm\,\Gamma\,[\sigma ; \tau] A} && {[ \sigma ]\;[ \tau ]\;t}
\end{alignat*}

We revisit \Cref{ex:pi2-comp} to demonstrate the pragmatic benefit of QIIRTs over QIITs.
\begin{example}
  The equation $\pi_2\,(\sigma; \tau) = [\sigma](\pi_2\,\tau)$ can be formally stated without any transport.
  The type of the term on the LHS is $[\pi_1(\sigma;\tau)]_{\Ty}\;A$ and is definitionally equal to $[\sigma]_\Ty\;([\pi_1\tau]_\Ty\;A)$, i.e.\ the type of the term on the RHS.
  Moreover, the first components in each step of the proof of \eqref{eq:pi2-comp-real-proof}, are definitionally equal because of \eqref{eq:def-type-subst-3} and \eqref{eq:def-type-subst-4}.
  Consequently, the back-of-the-envelope calculation~\eqref{eq:pi2-comp-proof} is \emph{formally correct}.
\end{example}

Yet, as we are defining a function on a quotient inductive type, we have to prove that $[\blank]_{\Ty}\blank$ is coherent.
This boils down to that $[\blank]_{\Ty}\blank$ preserves all equality constructors.
\LT{Should we mention general condition for coherence?} 
\begin{proposition}[Coherence]\label{prop:coherence-1}
  $[\sigma]_{\Ty}\,A$ is (propositionally) equal to $[\tau]_{\Ty}\,A$ for every equality constructor $p : \sigma = \tau$.
  \danger
\end{proposition}
\begin{proof}
  As equality constructors $[\idS]_T$, $[;]_T$, and $[]\UU$ for types become definitional, it suffices to show that following identities
  \begin{align*}
    [\sigma;\idS]_{\Ty} \;A          & = [\sigma]_{\Ty}\;A,
                                     & [\idS;\sigma]_{\Ty} \;A          & = [\sigma]_{\Ty}\;A,
                                     & [(\sigma;\tau);\gamma]_{\Ty} \;A & = [\sigma; (\tau; \gamma)]_{\Ty}A, \\
    [\pi_1(\sigma, t)]_{\Ty}\;A      & = [\sigma]_{\Ty}\;A, \\
    [\sigma; (\tau, t)]_{\Ty} \;A    & = [(\sigma;\tau), [\sigma]\, t]_{\Ty}\;A,
                                     & [\sigma]_{\Ty} \;A               & = [\emptyctx]_{\Ty}\;A,
                                     & [\sigma]_{\Ty} \;A               & = [\pi_1\sigma, \pi_2\sigma]_{\Ty}\;A
  \end{align*}
  hold (propositionally).
  The first two identities of the first row hold definitionally by \eqref{eq:def-type-subst-1} and \eqref{eq:def-type-subst-2}, while terms of the both sides of the third reduce to $[\sigma]_{\Ty}\;([\tau]_{\Ty}\;([\gamma]_{\Ty}\;A))$ by \eqref{eq:def-type-subst-2}.

  The sole identity of the second row is just one, i.e.\ \eqref{eq:def-type-subst-3}, of the clauses of the definition.

  We show the remaining cases by induction on $A$.
  However, $\UU$ is the only constructor for $\Ty\;\Gamma$ for any $\Gamma$ and each of identities with $A \defeq \UU$ reduce to $\UU$.
\end{proof}
\begin{remark}\label{re:coherence-proof}
  Note that in the above proof the first four identities hold definitionally by \eqref{eq:def-type-subst-1}--\eqref{eq:def-type-subst-4}, while the last three identities follow from \eqref{eq:def-type-subst-5} by induction on $A$.
  As we extend substitution calculus with other type formers, the first four identities will always hold but the last three depend on how congruence rules for type substitution is defined.
\end{remark}

\LT{We may wonder if the last three identities can be made definitional.
Can we find a complete set of rules?}

\subsection{... with an empty universe} \label{subsec:SC+U}
\LT[noinline]{Shall we go for Coquand universes? \cite{Coquand2013}}

In this section, we extend substitution calculus with an empty universe $\UU$ of small types~$\El\,u$.
For a QII definition \cite{Altenkirch2016a}, this extension amounts to adding the following constructors 
\begin{alignat*}{3}
  \UU   & : \implicit{\Gamma} && \Ty\, \Gamma \\
  \El   & : \implicit{\Gamma} && \Tm\,\Gamma\;\UU \to \Ty\,\Gamma \\
  []\El & : \implicit{\Gamma, \Delta, \sigma, u} && [ \sigma ]\,(\El\, u) =^{\Tm\;\Gamma\;\UU} \El\,(\alert{\transfib{\Tm\;\Gamma}{[]\UU}{\color{black}[\sigma]{u}}})
\end{alignat*}
where the type substitution $[\sigma](\El\,u)$ is, informally, the type of the term substitution $[\sigma]\,u$ for $u : \Tm\;\Delta\;\UU$, and the transport is needed to coerce $[\sigma]\,u : \Tm\;\Gamma\;([\sigma]\;\UU)$ into $\Tm\;\Gamma\;\UU$.

In our QIIR definition, $\transfib{\Tm\;\Gamma}{[]\UU}{\blank}$ is no longer needed, since the equality constructor $[]\UU$ becomes definitional. 
Therefore, the equality constructor $[]\El$ may be replaced by a clause of type substitution:
\begin{alignat}{3}
  [ \sigma ]_{\Ty}\,(\El\, u) & \reduce \El\,([\sigma] {u}) \label{eq:def-type-subst-6}
\end{alignat}
Yet, this naive change breaks the local confluence!
Observe that the local peak
\[
  \El\,u \Leftarrow [\idS]_{\Ty}(\El\,u) \reduce \El([\idS]\,u)
\]
cannot reduce to the same term, since the term substitution $[\idS]\,u$ is merely a constructor.

To repair the local confluence, we add a simultaneously defined function $[\blank]_{\Tm}\blank$ for term substitution apart from the explicit term substitution $[\blank]\blank$ as follows
\begin{alignat*}{3}
  [\blank]_{\Tm}\blank & : (\sigma : \Sub\,\Gamma\,\Delta) \to \Tm\,\Delta\,A \to \Tm\,\Gamma\,([\sigma]_{\Ty}\,A) \\
[ \idS ]_{\Tm}\,t          & \reduce t \\
[ \sigma ; \tau ]_{\Tm}\,t & \reduce [ \sigma ]_{\Tm}\;([ \tau ]_{\Tm}\;t) \\
[ \pi_1(\sigma, t) ]_{\Tm}\,t & \reduce [\sigma]_\Tm\,t \\
[ \pi_1(\sigma; \tau) ]_{\Tm}\,t & \reduce [\sigma]_\Tm\; ([\pi_1\tau]_\Tm \,t) \\
[ \sigma ]_{\Tm}\,t        & \reduce [ \sigma ]\,t, \quad \text{otherwise}
\end{alignat*}
which reduces substitutions that occur in \eqref{eq:def-type-subst-1}--\eqref{eq:def-type-subst-4} recursively so that the reduction can propagate from type to term substitution.
The remaining cases reduce to the explicit term substitution $[\sigma]\,t$.
Note that $[\blank]_{\Tm}\blank$ can be defined with the standard semantics for function definition, since there are no overlapping patterns.

Then, \eqref{eq:def-type-subst-6} and the rule $\mathsf{concat}$ are accordingly changed to 
\begin{align}
  [ \sigma ]_{\Ty}\,(\El\, u) & \reduce \El\,([\sigma]_{\Tm} {u}) \label{eq:def-type-subst-7}\\
  \mathsf{concat} : \sigma ; (\tau , t) & =^{\Sub\,\Gamma\,(\Theta, A)} (\sigma ; \tau) , [ \sigma ]_{\Tm} t \nonumber
\end{align}
to retain the local confluence for type substitution:
\begin{proposition}[Local confluence] \label{prop:local-confluence-2}
  Type substitution $[\blank]_{\Ty}\blank$ is locally confluent.
  \danger
\end{proposition}
\begin{proof}
  In addition to previous cases in the proof of \Cref{prop:local-confluence-1}, consider terms 
  \[
    [\idS]_{\Ty}\;(\El\,u), \qquad [\sigma;\tau]_{\Ty}\;(\El\,u), \qquad [\pi_1(\sigma, t)]_{\Ty}\;(\El\,u), \qquad\text{and}\qquad [\pi_1(\sigma; \tau)]_{\Ty}\;(\El\,u)
  \]
  which reduce to the following terms
  \[
    \El\,u, \qquad \El([\sigma]_{\Tm}\,[\tau]_{\Tm}\,u), \qquad \El([\sigma]_{\Tm}\,u), \qquad\text{and}\qquad
    \El([\sigma]_{\Tm}\,[\pi_1\tau]_{\Tm}\,u)
  \]
  respectively, no matter which function clause of $[\blank]_{\Ty}\blank$ is used.
  Therefore, type substitution is locally confluent.
\end{proof}
\begin{proposition}[Termination]
  Type substitution $[\blank]_{\Ty}\blank$ is terminating.
  \danger
\end{proposition}
\begin{proof}
  \LT{clearly terms on the RHS are structurally smaller~\cite{Abel2002}.}
\end{proof}

However, we also have to show that the recursive term substitution $[\blank]_{\Tm}\blank$ is propositionally equal to the explicit substitution $[\blank]\blank$, so they can be used together consistently.

\begin{proposition}\label{prop:correctness-1}
  For any $\sigma : \Sub\;\Gamma\;\Delta$, $A : \Ty\;\Delta$, and $t : \Tm\;\Delta\;A$, the term $[\sigma]t$ is propositionally equal to $[\sigma]_{\Tm}\,t$.
\end{proposition}
\begin{proof}
  We show this statement by induction on $\sigma$.
  \begin{enumerate}
    \item For the identity substitution $\idS$, we have $[\idS]t : [\idS]t = t$ but $t \judgeq [\idS]_{\Tm} t$ by definition.
    \item For a substitution composite $\sigma; \tau$, we have
      \begin{align*}
        [\sigma;\tau]t & = [\sigma]\,[\tau] t               && \text{by $[;]t$} \\
                       & = [\sigma]\,([\tau]_{\Tm}\,t)       && \text{by the induction hypothesis $[\tau]t = [\tau]_{\Tm}\,t$ for any $t$} \\
                       & = [\sigma]_{\Tm}\,([\tau]_{\Tm}\,t) && \text{by the induction hypothesis $[\sigma]u = [\sigma]_{\Tm}\,u$ for any $u$} \\
                       & \judgeq [\sigma;\tau]_{\Tm}\,t && \text{by definition.}
      \end{align*}
    \item For $\pi_1(\sigma, t)$, we have
      \begin{align*}
        [\pi_1(\sigma, t)]t & = [\sigma] t               && \text{by $\pi_1\beta$} \\
                            & = [\sigma]_{\Tm} \,t       && \text{by the induction hypothesis $[\tau]t = [\tau]_{\Tm}\,t$ for any $t$} \\
                            & \judgeq [\pi_1(\sigma, t)]_{\Tm}\,t && \text{by definition.}
      \end{align*}
    \item For $\pi_1(\sigma; \tau)$, it is proved similarly as the previous case.
    \item For the remaining cases, $[\sigma]_{\Tm} t$ is equal to $[\sigma]\,t$ definitionally.
  \end{enumerate}
  Hence, we conclude that $[\blank]_{\Tm}\blank$ is propositionally equal to $[\blank]\blank$.
  \LT[noinline]{Shall we say anything about equality constructors?}
\end{proof}

Finally, we have to check that both $[\blank]_{\Ty}\blank$ and $[\blank]_{\Tm}\blank$ are coherent with respect to equality constructors.
\begin{proposition}[Coherence] \label{prop:coherence-2}
  For every equality constructor $p : \sigma = \tau$, the following identities
  \danger
  \[
    [\sigma]_{\Ty}\,A = [\tau]_{\Ty}\,A
    \quad\text{and}\quad
    [\sigma]_{\Tm}\,t = [\tau]_{\Tm}\,t
  \]
  for any type $A$ and any term $t$.
\end{proposition}
\begin{proof}
  By \Cref{re:coherence-proof}, for type substitution, we only have to consider the following cases 
  \begin{align*}
    [\sigma; (\tau, t)]_{\Ty} \;A    & = [(\sigma;\tau), [\sigma]\, t]_{\Ty}\;A,
                                     & [\sigma]_{\Ty} \;A               & = [\emptyctx]_{\Ty}\;A,
                                     & [\sigma]_{\Ty} \;A               & = [\pi_1\sigma, \pi_2\sigma]_{\Ty}\;A
  \end{align*}
  while other cases hold definitionally.
  Again, for each case, we prove them by induction on $A$ and consider the case $A \defeq \El\,u$ only which is added to the definition of $[\blank]_{\Ty}\blank$ in this section. 
  However, all cases follow easily from \Cref{prop:correctness-1}.
  For example, consider the first case:
  \begin{align*}
    [\sigma; (\tau, t)]_{\Ty}(\El\,u) & \judgeq \El([\sigma; (\tau, t)]_{\Tm}\,u) && \text{by definition \eqref{eq:def-type-subst-7}} \\
                                      & = \El( [\sigma; (\tau, t)]\,u)            && \text{by \cref{prop:correctness-1}} \\
                                      & = \El( [\sigma; \tau, [\sigma]t)]\,u)     && \text{by $\mathsf{concat}$} \\
                                      & = \El( [\sigma; \tau, [\sigma]t)]_{\Tm}\,u) && \text{by \cref{prop:correctness-1}} \\
                                      & \judgeq [\sigma; \tau, [\sigma]t)]_{\Ty}\left(\El\,u\right) && \text{by definition \eqref{eq:def-type-subst-7} }
  \end{align*}
  The remaining two cases are omitted.

  As for term substitution $[\blank]_{\Tm}\blank$, each case follows from \cref{prop:correctness-1} in conjunction with the congruence rule for the identity type.
\end{proof}

We have seen that a naive extension of substitution calculus with a universe breaks the local confluence of type substitution defined in \cref{subsec:SC-QIIRT}, because the newly added function clause interacts with other existing clauses.
In general, every term occurs on the right-hand side of a clause needs to propagate the reduction to allow a confluent term to exist, and this desired propagation can be achieved by turning a term involving a substitution $\sigma$ into a recursion.
We will see the same pattern in the next section. 

\subsection{... and \texorpdfstring{$\Pi$}{Π}-types} \label{subsec:SC+U+Pi}

In this section, we extend the type theory in \cref{subsec:SC+U} with $\Pi$-types, i.e.\ dependent function types, whose the context of its codomain is extended with the domain type.
For the QIIT of type theory in \cite{Altenkirch2016a}, it amounts to add the following constructors for the type formation rule, the term introduction/elimination rules:
\begin{alignat*}{3}
  \Pi     &: \implicit{\Gamma}            && (A : \Ty\,\Gamma) \to \Ty\,(\Gamma, A) \to \Ty\,\Gamma \\
  \lambda &: \implicit{\Gamma, A, B}      && \Tm\;(\Gamma, A)\;B \to \Tm\,\Gamma\,(\Pi\;A\;B) \\
  \mathsf{app} &: \implicit{\Gamma, A, B} && \Tm\;\Gamma\;(\Pi\;A\;B) \to \Tm\;(\Gamma, A)\;B
\end{alignat*}
the computation rules
\begin{alignat*}{5}
  \Pi\beta           & : \implicit{\Gamma, A, B, t} && \mathsf{app}\,(\lambda\,t) && =^{\Tm\,\Gamma,A} && t \\
  \Pi\eta            & : \implicit{\Gamma, A, B, t} && t                        && =^{\Tm\,\Gamma,A} && \lambda(\mathsf{app}\,t)
\end{alignat*}
and the structural rules for substitution
\begin{alignat}{5}
  []\Pi              & : \implicit{\Gamma, \Delta, \sigma, A, B}    && [ \sigma ]_{\Ty}\,(\Pi\;A\;B) && =^{\Ty\;\Gamma} && \Pi\,(\sub{\sigma}{A})\,([\sigma\uparrow A]\,B) \label{eq:def-type-subst-8} \\
  \mathsf{[]\lambda} & : \implicit{\Gamma, \Delta, \sigma, A, B, t} && [\sigma]_{\Tm} (\lambda\,t) && =^{\Tm\,\Gamma}_{\alert{[]\Pi}} && \lambda\,([\sigma \uparrow A]\,t) \nonumber
\end{alignat}
where $\sigma \uparrow A$ is the \emph{lifting}  of $\sigma$ by a type $A$ defined as $(\pi_1\idS; \sigma, \alert{\transfib{\Tm\;\Gamma}{[;]_T}{\color{black}\pi_2\idS}})$. 

For our QIIR definition, $\sigma\uparrow A$ need not a transport since $[;]_T$ is definitional.
However, once again, adding \eqref{eq:def-type-subst-8} directly to the type substitution $[\blank]_{\Ty}\blank$ breaks the local confluence, because the reduction gets stuck on the lifting $\sigma \uparrow A$.
For example, $[\idS]_{\Ty}(\Pi\;A\;B)$ reduces to $\Pi\;A\;([\idS \uparrow A]_{\Ty}B)$ where $[\idS \uparrow A]_{\Ty}B$ is not definitionally equal to $B$, so the local peak $(\Pi\;A\;B) \Leftarrow [\idS]_{\Ty}(\Pi\;A\;B) \reduce \Pi\;A\;([\idS \uparrow A]_{\Ty}B)$ cannot be joined by a confluent term. 

To retain the local confluence, we turn $\sigma \uparrow A$ into a simultaneously defined function:
\begin{alignat}{3}
  \blank\uparrow \blank & : (\sigma : \Sub\,\Gamma,\Delta) \to (A : \Ty\,\Delta) \to \Sub\,(\Gamma, [\sigma]_{\Ty}\,A)\, (\Delta, A) \label{eq:type-of-lifting} \\
\idS                \uparrow A & \reduce \idS \nonumber \\
(\sigma ; \tau)     \uparrow A & \reduce (\sigma \uparrow \sub{\tau}{A} ) ; (\tau \uparrow A) \nonumber \\
\pi_1(\sigma, t)    \uparrow A & \reduce \sigma \uparrow A \nonumber \\
\pi_1(\sigma; \tau) \uparrow A & \reduce (\sigma \uparrow (\sub{\pi_1 \tau}{A})) ; (\pi_1 \tau \uparrow A)\nonumber \\
\sigma              \uparrow A & \reduce \sigma^+ , \quad \text{otherwiese} \nonumber
\end{alignat}
where $\sigma^+ \defeq (\pi_1\idS ; \sigma, \pi_2\idS)$.
Then, we add another clause for type substitution and an equality constructor for term substitution without transport:
\begin{alignat}{5}
                     & [ \sigma ]_{\Ty}\,(\Pi\,A\,B) && \reduce \Pi\,(\sub{\sigma}{A})\,(\sub{\sigma\uparrow A}{B}) \label{eq:def-type-subst-9} \\
  \mathsf{[]\lambda} & : [\sigma]_{\Tm} (\lambda\,t) && =^{\Tm\,\Gamma\,([\sigma]_{\Ty}(\Pi\,A\,B))} && \lambda\,([\sigma \uparrow A]_{\Tm}\,t) \nonumber
\end{alignat}

\begin{proposition}[Local confluence]\label{prop:local-confluence-3}
  Type substitution $[\blank]_{\Ty}\blank$ is locally confluent.
  \danger
\end{proposition}
\begin{proof}
  In addition to previous cases in \cref{prop:local-confluence-1,prop:local-confluence-2}, consider
  \begin{align*}
    [\idS]_{\Ty}\;(\Pi\;A\;B) && [\sigma;\tau]_{\Ty}\;(\Pi\;A\;B) && [\pi_1(\sigma, t)]_{\Ty}\;(\Pi\;A\;B) && [\pi_1(\sigma; \tau)]_{\Ty}\;(\Pi\;A\;B).
  \end{align*}
  Each of them reduces to the following terms after applying clauses of $\blank\uparrow\blank$ and $[\blank]_{\Ty}\blank$:
  \begin{align*}
    & \Pi\;A\;B                && \Pi\;([\sigma]_{\Ty}[\tau]_{\Ty}\,A)\;([\sigma \uparrow [\tau]\,A ]_{\Ty}[\tau \uparrow A ]_{\Ty}\,B) \\
    & \Pi\;([\sigma]_{\Ty}\;A)([\sigma \uparrow A]_{\Ty}\;B) && \Pi\;([\sigma]_{\Ty}\,[\pi_1\tau]_{\Ty}\,A)\;([\sigma \uparrow [\pi_1\tau]_{\Ty}[\pi_1\tau \uparrow A]_{\Ty}\,B)
  \end{align*}
  respectively.
  Hence, the type substitution $[\blank]_{\Ty}\blank$ is locally confluent.
\end{proof}
\begin{proposition}[Termination]
  Type substitution $[\blank]_{\Ty}\blank$ is terminating.
  \danger
\end{proposition}

Similar to term substitution, we can show that the lifted substitution $\sigma \uparrow A$ is propositionally equal to $\sigma^+$, based on two basic properties\footnote{%
  The reader is invited to prove them using the QII definition of type theory.}
about the lifting.
\begin{lemma} \label{lem:lifting}
  The following statements hold:
  \begin{enumerate}
    \item $\idS\;\{\Gamma, A\} = (\idS\;\{\Gamma\})^+$ and
    \item $(\sigma; \tau)^+ = \sigma^+ ; \tau^+$.
  \end{enumerate}
\end{lemma}

\begin{proposition} \label{prop:correctness-2}
  For every $\sigma : \Sub\;\Gamma\;\Delta$, and $A : \Ty\;\Delta$
  $\sigma \uparrow A$ is propositionally equal to $(\pi_1 \idS ; \sigma , \pi_2 \idS)$.
\end{proposition}
\begin{proof}
  We prove the statement by induction on $\sigma$ with \cref{lem:lifting}.
  Every case follows from equational reasoning straightforwardly.
  For example, if $\sigma = \sigma;\tau$ is a composite, we have
  \begin{align*}
    (\sigma;\tau) \uparrow A & \judgeq (\sigma \uparrow [\tau]\,A);(\tau \uparrow A) && \text{by definition} \\
                             & = \sigma^+;\tau^+                                     && \text{by induction hypothesis} \\ 
                             & = (\sigma;\tau)^+                                     && \text{by \cref{lem:lifting}.} 
  \end{align*}
  Other cases follow similarly.
\end{proof}

As we are working with a different type theory from \cref{subsec:SC+U}, we still have to prove the propositional equality $[\sigma]\,t = [\sigma]_{\Tm}\,t$ but the same argument still applies:
\begin{proposition} \label{prop:correctness-3}
  For every $\sigma : \Sub\;\Gamma\;\Delta$, $A : \Ty\;\Delta$, and $t : \Ty\;\Delta\;A$,
  the term $[\sigma]t$ is propositionally equal to $[\sigma]_{\Tm}\,t$.
\end{proposition}

Finally, we show that these simultaneously defined functions are coherent.
\begin{proposition}[Coherence]
  For every equality constructor $p : \sigma = \tau$, the following identities
  \danger
  \[
    \sigma \uparrow A = \tau \uparrow A
    \qquad\text{and}\qquad
    [\sigma]_{\Tm}\,t = [\tau]_{\Tm}\,t,
  \]
  hold for any type $A$ and any term $t$.
\end{proposition}
\begin{proof}
  These two identities follow from \cref{prop:correctness-2,prop:correctness-3} easily.
\end{proof}

\subsubsection{Multiple lifting and the coherence of type substitution}
For the coherence of~$[\blank]_{\Ty}\blank$, we have to generalise the desired identity to account for the lifting, since a naive induction does not work: for example, unfolding $[\sigma]_{\Ty}\,(\Pi\;A\;B)$ leads to
\[
  \Pi\;([\sigma]_{\Ty}\,A)\;([\sigma\uparrow A]_{\Ty}\,B)
\]
where the induction hypothesis $[\sigma]_{\Ty}\,B = [\emptysub]_{\Ty}\,B$ cannot be applied.
Instead, we will define the \emph{multiple lifting} $\sigma \upuparrows \Xi$ by a telescope $\Xi = A_1, \dots, A_n$, which intuitively means multiple applications of lifting: $\sigma \uparrow A_1 \uparrow \dots \uparrow A_n$.
Then, we show the identity
$[\sigma \mathop{\upuparrows} \Xi ]_{\Ty}\,A = [\tau \mathop{\upuparrows} \Xi ]_{\Ty}\,A$
for arbitrary $\Xi$, subsuming the special case $[\sigma ]_{\Ty}\,A = [\tau ]_{\Ty}\,A$ for $\Xi$ being empty.

First, define an inductive-recursive type of \emph{telescopes} under a context $\Gamma$ simultaneously with a concatenation $\blank \dplus \blank$ as follows.
\begin{alignat*}{5}
  \Tel               & :                   && \Ctx \to \Set                                     \\
  \blank\dplus\blank & :                   && (\Gamma : \Ctx) \to (\Xi : \Tel\;\Gamma) \to \Ctx \\
  \emptytel          & : \implicit{\Gamma} && \Tel\;\Gamma \\
  \blank,\blank      & : \implicit{\Gamma} && (\Xi : \Tel\;\Gamma) \to (A : \Ty\;(\Gamma \dplus \Xi)) \to \Tel\;\Gamma \\
  \Gamma \dplus \emptytel & = && \Gamma \\
  \Gamma \dplus (\Xi, A)  & = && (\Gamma \dplus \Xi) , A
\end{alignat*}

Then, we define the multiple lifting $\blank \upuparrows \blank$ which generalises the single lifting $\blank\uparrow\blank$.
Observe that its type, generalising the type \eqref{eq:type-of-lifting} of a single lifting $\blank\uparrow\blank$, should be
\[
\blank\upuparrows \blank : (\sigma : \Sub\;\Gamma\;\Delta) \to (\Xi : \Tel\;\Delta)
\to \Sub\;(\Gamma \dplus ([\sigma]_{\Tel}\,\Xi))\;(\Delta \dplus \Xi)
\]
requiring telescope substitution $[\blank]_{\Tel}\blank :  \Sub\;\Gamma\;\Delta \to \Tel\;\Delta \to \Tel\;\Gamma$, but $[\blank]_{\Tel}\blank$ also needs the multiple lifting for the case $[ \sigma ]_{\Tel}\,(\Xi, A)$ (why?).
Hence, we arrive the conclusion that telescope substitution and the multiple lifting by a telescope need to be defined mutually:
\begin{alignat*}{5}
  & [ \sigma ]_{\Tel}\,\emptytel && = \emptytel \\
  & [ \sigma ]_{\Tel}\,(\Xi, A)  && = [ \sigma ]_{\Tel}\,\Xi, [\sigma \upuparrows \Xi]_{\Ty}\,A \\
  & \sigma \upuparrows \emptytel && = \sigma \\
  & \sigma \upuparrows (\Xi, A)  && = (\sigma \upuparrows \Xi), A
\end{alignat*}
Now, we have to prove the coherence property for the multiple lifting, telescope substitution, and type substitution altogether.
\begin{proposition}
  For every equality constructor $p : \sigma = \tau$, the identities
  \danger
  \[
    \sigma \upuparrows \Xi = \tau \upuparrows \Xi, \quad
    [\sigma]_{\Tel}\,\Xi = [\tau]_{\Tel}\,\Xi,
    \quad\text{and}\quad
    [\sigma \upuparrows \Xi ]_{\Ty}\,A = [\tau \upuparrows \Xi ]_{\Ty}\,A
  \]
  hold for any telescope $\Xi$ and type $A$.
\end{proposition}
\begin{proof}
  We prove these two identities together by induction on the telescope $\Xi$ for the first identity and the type $A$ for the second.

  \LT{
  By \cref{re:coherence-proof}, we consider the following case
  \begin{align*}
    [\sigma; (\tau, t)]_{\Ty} \;A    & = [(\sigma;\tau), [\sigma]\, t]_{\Ty}\;A,
                                     & [\sigma]_{\Ty} \;A               & = [\emptyctx]_{\Ty}\;A,
                                     & [\sigma]_{\Ty} \;A               & = [\pi_1\sigma, \pi_2\sigma]_{\Ty}\;A
  \end{align*}
  We only consider the case $A = \Pi\;A\;B$ in addition to those considered in \cref{prop:coherence-2}.
  }
\end{proof}


%% ⟨_⟩ : Tm Γ A → Sub Γ (Γ , A)
%% ⟨ t ⟩ = idS , t
%% 
%% _$$_
%%   : (t : Tm Γ (Π A B)) (u : Tm Γ A)
%%   → Tm Γ ([ ⟨ u ⟩ ] B)
%% t $$ u = [ ⟨ u ⟩ ]t app t

We conclude this extension with $\Pi$-types by noting that single term substitution
\begin{alignat*}{3}
  \left<\blank\right> & : \Tm\;\Gamma\;A \to \Sub\;\Gamma\;(\Gamma, A) \\
  \left< t \right> & \defeq (\idS , t)
\end{alignat*}
can be introduced without any transport, which is $(\idS, \alert{\transfib{\Tm\;\Gamma}{[id]_T^{-1}}{\color{black}t}})$ in its QII definition, and which can used to define the ordinal application: $t \mathop{\$} u \defeq [ \left< u \right> ]_{\Tm} (\mathsf{app}\,t)$.
Similarly, the structural rule for substitution on $\mathsf{app}$ can be derived
\begin{align*}
  [\sigma \uparrow A ]_{\Tm}(\mathsf{app}\;t) & = \mathsf{app}\,\lambda([\sigma \uparrow A ]_{\Tm}(\mathsf{app}\;t)) && \text{by $\Pi\beta^{-1}$} \\
                                              & = \mathsf{app}\,\left([\sigma]\,\lambda\,(\mathsf{app}\,t)\right) && \text{by $[]\lambda^{-1}$} \\
                                              & = \mathsf{app}\,([\sigma]\,t) && \text{by $\Pi\eta^{-1}$.}
\end{align*}
without any transport (cf.\ \cite{Altenkirch2016a}), since \eqref{eq:def-type-subst-8} has become definitional. 


\subsection{... and other type formers} \label{subsec:SC+U+Pi+more}
\LT{%
\begin{enumerate}
  \item Discuss the unit type.
  \item Discuss the Boolean type.
  \item Discuss the type of natural numbers.
  \item Discuss the extensional identity type.
\end{enumerate}
}

\section{The eliminator and the standard model}
\LT{%
\begin{enumerate}
  \item Introduce motives and methods for the definition in \Cref{subsec:SC+U+Pi}.
  \item The rule $\mathsf{Elim}_{\Ty}\,([ \sigma ]_{\Ty}\,A) = [ \mathsf{Elim}_{\Sub}\,\sigma ]^{A}_{\Ty}(\mathsf{Elim}_{\Ty}\,A)$ is not defined but provable. 
  \item $\mathsf{Elim}_{\Tm}\,([ \sigma ]_{\Tm}\,t) = [ \mathsf{Elim}_{\Sub}\,\sigma ]^{A}_{\Tm}(\mathsf{Elim}_{\Tm}\,t)$ is proved.
\end{enumerate}
}

\section{Equivalence to the quotient inductive-inductive definition of type theory}
\LT{%
\begin{enumerate}
  \item Show that two definitions are equivalent (ongoing).
  \item Show that methods are equivalent (ongoing).
\end{enumerate}
}

\section{Discussion and future work}
\subsection{Related work}
\paragraph*{Formalisation of type theory in type theory}
\cite{Danielsson2006,Altenkirch2016a,Chapman2009}
\cite{Altenkirch2017}
\cite{Munoz1998}
\cite{Dybjer1996,Castellan2021}
\LT{Dybjer \cite{Dybjer1996} already says cwf is inductive-recursive}

Substitution calculus is closely related to \emph{categories with families}~\cite{Dybjer1996}.
\paragraph*{Single substitution calculus}
\cite{Kaposi2023,Kaposi2024a}
\paragraph*{Schemata of inductive types}
\cite{Kaposi2019}


\subsection{Future work}
\paragraph*{Type theory in (observational) type theory}
\paragraph*{Formalising definitions by overlapping patterns}
\paragraph*{A general schema of quotient inductive-inductive-recursive types}

\IfFileExists{./reference.bib}{\bibliography{reference}}{\bibliography{ref}}

\appendix

\section{Complete definitions}
\begin{alignat*}{3}
  \Ctx      & : && \Set                   \\
  \Ty       & : && \Ctx \to \Set          \\
  \Sub      & : && \Ctx \to \Ctx \to \Set \\
  \Tm       & : && (\Gamma : \Ctx) \to \Ty\,\Gamma \to \Set \\
  \emptyctx & : && \Ctx \\
  \blank,\blank & : && (\Gamma : \Ctx) \to \Ty\,\Gamma \to \Ctx \\
  [\blank]\blank & : \implicit{\Gamma, \Delta} && \Sub\,\Gamma\,\Delta \to \Ty\,\Delta \to \Ty\,\Gamma \\
  \emptysub & : \implicit{\Gamma} && \Sub\,\Gamma\,\emptyctx \\
  \blank,\blank & : \implicit{\Gamma, \Delta, A} && (\sigma : \Sub\,\Gamma\,\Delta) \to \Tm\,\Gamma\,([ \sigma ]_{\Ty} A) \to \Sub\,\Gamma,(\Delta, A) \\
  \idS & : \implicit{\Gamma} && \Sub\,\Gamma\,\Gamma \\
  \blank;\blank & : \implicit{\Gamma, \Delta, \Theta} && \Sub\,\Gamma\,\Delta \to \Sub\,\Delta\,\Theta \to \Sub\,\Gamma\,\Theta \\
  \pi_1 & : \implicit{\Gamma, \Delta, A} && \Sub\,\Gamma\,(\Delta, A) \to \Sub\,\Gamma\,\Delta \\
  \pi_2 & : \implicit{\Gamma, \Delta, A} && (\sigma : \Sub\,\Gamma\,(\Delta, A)) \to \Tm\,\Gamma\,([ \pi_1\,\sigma ]\, A) \\
  [\blank] \blank & : \implicit{\Gamma,\Delta, A} && (\sigma : \Sub\,\Gamma\,\Delta) \to \Tm\,\Delta\,A \to \Tm\,\Gamma\,([\sigma]\, A) \\
  \UU     & : \implicit{\Gamma} && \Ty\, \Gamma \\
  \El     & : \implicit{\Gamma} && \Tm\,\Gamma\,U \to \Ty\,\Gamma \\
  \Pi     & : \implicit{\Gamma} && (A : \Ty\,\Gamma) \to \Ty\,(\Gamma, A) \to \Ty\,\Gamma \\
\end{alignat*}

\begin{alignat*}{5}
[ \idS ]_{\Ty}\,A             & \reduce A \\
[ \sigma ; \tau ]_{\Ty}\,A    & \reduce [ \sigma ]_{\Ty}\;([ \tau ]_{\Ty}\;A) \\
[ \pi_1(\sigma, t) ]_{\Ty}\,A & \reduce [\sigma]_\Ty\,A \\
[ \pi_1(\sigma; \tau) ]_{\Ty}\,A & \reduce [\sigma]_\Ty ([\pi_1\tau]_\Ty \,A) \\
[ \sigma ]_{\Ty}\,\UU           & \reduce \UU \\
[ \sigma ]_{\Ty}\,(\El\, u) & \reduce \El\,([\sigma]_{\Tm}{u}) \\
[ \sigma ]_{\Ty}\,(\Pi\,A\,B) & \reduce \Pi\,(\sub{\sigma}{A})\,(\sub{\sigma\uparrow A}{B}) \\
\idS                \uparrow A & \reduce \idS \\
\sigma ; \tau       \uparrow A & \reduce (\sigma \uparrow \sub{\tau}{A} ) ; (\tau \uparrow A) \\
\pi_1(\sigma, t)    \uparrow A & \reduce \sigma \uparrow A \\
\pi_1(\sigma; \tau) \uparrow A & \reduce \sigma \uparrow (\sub{\pi_1 \tau}{A}) ; (\pi_1 \tau \uparrow A) \\
\sigma              \uparrow A & \reduce (\pi_1 \idS ; \sigma , \pi_2 \idS) & \text{otherwiese} \\
 [ \idS ]_{\Tm}\,t                & \reduce t \\
 [ \sigma ; \tau ]_{\Tm}\,t       & \reduce [ \sigma ]_{\Tm}\;([ \tau ]_{\Tm}\;t) \\
 [ \pi_1(\sigma, t) ]_{\Tm}\,t    & \reduce [\sigma]_\Tm\,t \\
 [ \pi_1(\sigma; \tau) ]_{\Tm}\,t & \reduce [\sigma]_\Tm\; ([\pi_1\tau]_\Tm \,t) \\
 [ \sigma ]_{\Tm}\,t              & \reduce [ \sigma ]\,t & \text{otherwise}
\end{alignat*}

\begin{alignat*}{5}
  \mathsf{idr}    & : \implicit{\Gamma, \Delta, \sigma} && {\sigma ; \idS_{\Delta}} && =^{\Sub\,\Gamma\,\Delta} && {\sigma} \\
  \mathsf{idl}    & : \implicit{\Gamma, \Delta, \sigma} && {\idS_{\Gamma} ; \sigma} && =^{\Sub\,\Gamma\,\Delta} && {\sigma} \\
  ;\text{-}\mathsf{assoc} & : \implicit{\Gamma, \Delta, \Theta, \sigma, \tau, \gamma} && (\sigma ; \tau) ; \gamma && =^{\Sub\,\Gamma\,\Theta} &&  \sigma ; (\tau ; \gamma) \\
  \mathsf{concat} & : \implicit{\Gamma, \Delta, \Theta, \sigma, \tau, A, t} &&\sigma ; (\tau , t)      && =^{\Sub\,\Gamma\,(\Theta, A)} &&  (\sigma ; \tau) , [ \sigma ] t \\
  \emptyctx\eta   & : \implicit{\Gamma, \sigma} && \sigma                   && =^{\Sub\,\Gamma\,\emptyctx} & \emptysub \\
  \pi\eta         & : \implicit{\Gamma, A, \sigma} && \sigma                   && =^{\Sub\,\Gamma\,(\Delta, A)} &&  (\pi_1 \sigma, \pi_2 \sigma) \\
  \pi_1\beta      & : \implicit{\Gamma, \Delta, \Theta, \sigma, A, t} && \pi_1(\sigma , t)        && =^{\Sub\,\Gamma,\Delta} &&  \sigma \\
  \pi_2\beta      & : \implicit{\Gamma, \Delta, \Theta, \sigma, A, t} && \pi_2(\sigma , t)        && =^{\Tm\,\Gamma\, A} &&  t \\
  [\idS]t         & : \implicit{\Gamma, A, t} && {[\,\idS\,]\,t}          && =^{\Tm\,\Gamma\,A} && t \\
  [;]t            & :\implicit{\Gamma, \Delta, \Theta, \sigma, \tau, t} && {[\,\sigma ; \tau\,]\,t} && =^{\Tm\,\Gamma\,[\sigma ; \tau] A} && {[ \sigma ]\,[ \tau ]\,t} \\
\end{alignat*}

\section{Formal definitions in Agda}
\end{document}
